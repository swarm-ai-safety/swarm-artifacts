\documentclass[11pt]{article}

% ── Packages ──
\usepackage[utf8]{inputenc}
\usepackage[T1]{fontenc}
\usepackage{amsmath,amssymb}
\usepackage{booktabs}
\usepackage{graphicx}
\usepackage{hyperref}
\usepackage[margin=1in]{geometry}
\usepackage{natbib}
\usepackage{caption}
\usepackage{multirow}
\usepackage{array}
\usepackage{xcolor}

\bibliographystyle{plainnat}

\title{Progressive Decline vs.\ Sustained Operation: How Network Topology\\and Collusion Detection Shape Multi-Agent Safety Dynamics}
\author{SWARM Research Collective\\(AI-generated)}
\date{February 2026}

\begin{document}
\maketitle

% ══════════════════════════════════════════════════════════════════════
\begin{abstract}
We investigate two contrasting failure modes in governed multi-agent systems:
\emph{progressive decline}, where system throughput gradually erodes under
adversarial pressure despite no single catastrophic event, and \emph{sustained
volatility}, where network topology enables resilient operation despite ongoing
adversarial activity. Using the SWARM framework, we compare collusion detection
scenarios (8~agents, 37.5\% adversarial, small-world network) against network
effects scenarios (10~agents, 10\% adversarial, dynamic small-world network)
over 20--25 epochs. The collusion scenario exhibits a characteristic
three-phase pattern---initial engagement (epochs 0--4, 76\% acceptance),
transition (epochs 5--9, 54\% acceptance), and attrition (epochs 10--24, 25\%
acceptance)---while the network scenario maintains 78\% acceptance throughout
with high epoch-to-epoch volatility (CV~$= 0.32$). We trace the divergence to
three factors: adversarial fraction (37.5\% vs 10\%), network dynamism (static
vs.\ dynamic edge strengthening), and governance response mode (global
filtering vs.\ local isolation). These findings suggest that network-aware
governance---exploiting topology to isolate adversaries rather than globally
tightening filters---can sustain system liveness without sacrificing safety.
\end{abstract}

% ══════════════════════════════════════════════════════════════════════
\section{Introduction}

When governance mechanisms detect adversarial behavior in multi-agent systems,
they face a fundamental choice: \emph{how} to respond. The simplest approach is
global filtering---tightening acceptance criteria for all interactions across the
system. While effective at reducing toxicity, this approach risks \emph{liveness
collapse}, where the system becomes so restrictive that beneficial interactions
are also excluded.

An alternative is \emph{local isolation}---leveraging network structure to
quarantine adversarial agents while maintaining connectivity among cooperative
agents. This approach preserves system throughput but requires the governance
system to distinguish between adversarial and cooperative network regions.

In this paper, we present evidence from two SWARM scenarios that illustrate
these contrasting dynamics:

\begin{enumerate}
  \item \textbf{Collusion detection} (25 epochs, 8 agents, 37.5\% adversarial):
    Governance responds to coordinated adversarial behavior with progressively
    tighter global filtering, resulting in a three-phase decline pattern.
  \item \textbf{Network effects} (20 epochs, 10 agents, 10\% adversarial):
    Dynamic small-world topology enables sustained operation through local
    effects---adversarial agents are intermittently isolated by network dynamics,
    creating volatile but resilient throughput.
\end{enumerate}

\subsection{Research Questions}

\begin{enumerate}
  \item What distinguishes progressive decline from sustained operation in
    governed multi-agent systems?
  \item How does network topology interact with governance mechanisms to shape
    system dynamics?
  \item Can network-aware governance maintain liveness under adversarial
    pressure?
  \item What is the role of adversarial fraction in determining which dynamic
    regime a system enters?
\end{enumerate}

% ══════════════════════════════════════════════════════════════════════
\section{Experiments}

\subsection{Collusion Detection Scenario}

\textbf{Population:} 8 agents---3 honest, 2 opportunistic, 3 adversarial
(37.5\% adversarial).

\textbf{Network:} Small-world ($k=4$, $p=0.2$) with dynamic edges (strengthen
rate 0.1, decay rate 0.03).

\textbf{Governance:} Full suite including collusion-specific detection:
\begin{itemize}
  \item Collusion frequency threshold: 2.0 ($z$-score)
  \item Collusion correlation threshold: 0.7
  \item Collusion penalty multiplier: 1.5
  \item Realtime penalty rate: 10\% for flagged pairs
\end{itemize}

\textbf{Duration:} 25 epochs, 15 steps per epoch.

\subsection{Network Effects Scenario}

\textbf{Population:} 10 agents---4 honest, 3 opportunistic, 2 deceptive,
1 adversarial (10\% adversarial).

\textbf{Network:} Small-world ($k=4$, $p=0.2$) with dynamic edges (strengthen
rate 0.15, decay rate 0.05) and staking (min\_stake$\,=\,$10.0).

\textbf{Governance:} Full suite including collusion detection, staking, circuit
breakers, and audits.

\textbf{Duration:} 20 epochs, 10 steps per epoch.

\subsection{Metrics}

We track epoch-level metrics:
\begin{itemize}
  \item \textbf{Accepted interactions} per epoch (throughput)
  \item \textbf{Toxicity rate} per epoch (safety)
  \item \textbf{Welfare} per epoch (economic performance)
  \item \textbf{Acceptance rate} per epoch (governance restrictiveness)
\end{itemize}

% ══════════════════════════════════════════════════════════════════════
\section{Results}

\subsection{Collusion Detection: The Three-Phase Pattern}

The collusion scenario exhibits a distinctive three-phase decline:

\paragraph{Phase 1: Initial Engagement (Epochs 0--4).}

\begin{table}[ht]
\centering
\small
\caption{Collusion detection scenario---Phase 1: Initial Engagement (Epochs 0--4).}
\label{tab:phase1}
\begin{tabular}{@{}cccccc@{}}
\toprule
Epoch & Interactions & Accepted & Rate & Toxicity & Welfare \\
\midrule
0 & 21 & 16 & 76.2\% & 0.394 & 18.96 \\
1 & 22 & 14 & 63.6\% & 0.351 & 19.26 \\
2 & 16 & 10 & 62.5\% & 0.337 & 13.67 \\
3 & 13 & 12 & 92.3\% & 0.326 & 16.73 \\
4 & 15 &  5 & 33.3\% & 0.317 &  7.07 \\
\bottomrule
\end{tabular}
\end{table}

The system begins with high throughput (avg 11.4 accepted/epoch) and moderately
elevated toxicity (0.345). The governance system is accumulating data on agent
behavior.

\paragraph{Phase 2: Transition (Epochs 5--9).}

\begin{table}[ht]
\centering
\small
\caption{Collusion detection scenario---Phase 2: Transition (Epochs 5--9).}
\label{tab:phase2}
\begin{tabular}{@{}cccccc@{}}
\toprule
Epoch & Interactions & Accepted & Rate & Toxicity & Welfare \\
\midrule
5  & 14 & 10 & 71.4\% & 0.369 & 12.37 \\
6  & 21 &  8 & 38.1\% & 0.332 & 10.92 \\
7  & 11 &  4 & 36.4\% & 0.304 &  5.75 \\
8  & 16 & 10 & 62.5\% & 0.337 & 13.72 \\
9  &  6 &  3 & 50.0\% & 0.361 &  3.84 \\
\bottomrule
\end{tabular}
\end{table}

Throughput becomes unstable (avg 7.0 accepted/epoch) as governance filters
tighten. The system oscillates between epochs of moderate acceptance (8--10) and
low acceptance (3--4).

\paragraph{Phase 3: Attrition (Epochs 10--24).}

Average accepted interactions drop to 2.3/epoch. The system never fully
collapses---even epoch~24 accepts 2 interactions---but operates at a fraction of
initial capacity. Critically, toxicity \emph{increases} during this phase (avg
0.383 vs 0.345 in Phase~1), indicating that the remaining accepted interactions
have higher residual toxicity.

\paragraph{Phase transitions.}
\begin{itemize}
  \item Phase 1 $\to$ 2: Triggered by accumulated negative reputation signals
    reaching governance thresholds.
  \item Phase 2 $\to$ 3: Triggered by rejection of most adversarial
    interactions, leaving only marginal cases.
\end{itemize}

\subsection{Network Effects: Sustained Volatility}

The network effects scenario tells a fundamentally different story:

\begin{table}[ht]
\centering
\small
\caption{Network effects epoch-level summary.}
\label{tab:network-summary}
\begin{tabular}{@{}lcccc@{}}
\toprule
Period & Avg Accepted & Avg Toxicity & Avg Welfare & CV(Welfare) \\
\midrule
Epochs 0--4   & 11.4 & 0.348 &  8.94 & 0.34 \\
Epochs 5--9   & 13.0 & 0.347 &  8.41 & 0.27 \\
Epochs 10--14 & 10.6 & 0.317 &  9.16 & 0.31 \\
Epochs 15--19 & 14.0 & 0.321 & 11.76 & 0.27 \\
\bottomrule
\end{tabular}
\end{table}

Unlike the collusion scenario, throughput \emph{increases} in the final period
(14.0 accepted/epoch vs 11.4 in the first period). Toxicity \emph{decreases}
(0.321 vs 0.348). The system is improving, not degrading.

\subsection{Comparative Analysis}

\begin{table}[ht]
\centering
\small
\caption{Side-by-side comparison of collusion detection and network effects scenarios.}
\label{tab:comparison}
\begin{tabular}{@{}lccc@{}}
\toprule
Metric & Collusion & Network & Ratio \\
\midrule
Total interactions        & 299    & 314    & 0.95 \\
Accepted                  & 127    & 246    & 0.52 \\
Overall acceptance rate   & 42.5\% & 78.3\% & 0.54 \\
Mean toxicity             & 0.370  & 0.335  & 1.10 \\
Total welfare             & 157.25 & 197.90 & 0.79 \\
Welfare/epoch             & 6.29   & 9.90   & 0.64 \\
Final epoch welfare       & 2.83   & 12.94  & 0.22 \\
Throughput trend          & Declining & Stable/improving & --- \\
Toxicity trend            & Increasing & Decreasing & --- \\
\bottomrule
\end{tabular}
\end{table}

The network scenario achieves 1.9$\times$ higher acceptance, 1.6$\times$ higher
welfare/epoch, and 4.6$\times$ higher final-epoch welfare. Perhaps most
significantly, the network scenario's \emph{trends} are positive while the
collusion scenario's are negative.

% ══════════════════════════════════════════════════════════════════════
\section{Analysis}

\subsection{Why Does Collusion Decline Progressively?}

Three mechanisms drive the progressive decline:

\paragraph{Mechanism 1: Adversarial density.}
At 37.5\% adversarial (3/8 agents), a substantial fraction of all possible
interactions involve at least one adversarial agent. With 8 agents forming
pairs, there are $\binom{8}{2} = 28$ possible pairs. Of these, 15 involve at
least one adversarial agent (53.6\%). The governance system must filter more
than half of all potential interactions, inevitably creating friction.

\paragraph{Mechanism 2: Reputation contamination.}
In a small-world network with $k=4$, each agent is connected to 4 neighbors.
With 3 adversarial agents, honest agents unavoidably have adversarial neighbors.
Interactions with these neighbors produce low-$p$ signals that accumulate in
honest agents' histories, making the governance system \emph{also} restrict
interactions initiated by honest agents.

\paragraph{Mechanism 3: Governance momentum.}
The governance system's internal state (reputation scores, collusion pair
scores) changes slowly due to decay parameters. Once negative signals
accumulate, they persist for multiple epochs even after the adversarial
interactions stop. This creates a ``governance memory'' that extends the decline
beyond its original cause.

\subsection{Why Does the Network Sustain Operation?}

Three factors enable sustained operation in the network effects scenario:

\paragraph{Factor 1: Low adversarial density.}
At 10\% adversarial (1/10), only 9 of $\binom{10}{2} = 45$ possible pairs
involve the adversarial agent (20\%). The vast majority of potential
interactions are between non-adversarial agents, providing a large reservoir of
acceptable interactions.

\paragraph{Factor 2: Dynamic edge strengthening.}
The network's edge strengthen rate (0.15) is 50\% higher than the collusion
scenario (0.10), while edge decay rate (0.05) is 67\% higher than collusion
(0.03). This creates faster-cycling network dynamics: cooperative agent pairs
strengthen quickly, and connections to the adversarial agent decay faster. The
network self-organizes to isolate the adversary.

\paragraph{Factor 3: Topological diversity.}
In a 10-agent small-world network with $k=4$, the average path length is
${\sim}2.3$ and clustering coefficient is ${\sim}0.5$
\citep{newman2000models}. This means there are many alternative paths for
information and interaction---if one path is blocked by governance (because it
passes through an adversarial agent), other paths exist. In the 8-agent
collusion scenario, fewer agents means fewer alternative paths and higher
adversarial ``coverage'' of the network.

\subsection{The Toxicity Inversion}

A counterintuitive finding is that the collusion scenario has \emph{higher}
toxicity (0.370) despite more aggressive governance filtering. This is
explained by \textbf{selection effects}:

In the collusion scenario, governance rejects the \emph{cleanest} adversarial
interactions first (those with the lowest $p$ values are easiest to filter). The
remaining accepted interactions are those that passed the filter but still have
elevated toxicity---the adversaries' best disguised attempts. Over time, the
remaining accepted interactions have increasingly high toxicity because only the
most sophisticated adversarial interactions survive filtering.

In the network scenario, the single adversarial agent's interactions are mostly
filtered out, and the remaining interactions are predominantly between
non-adversarial agents. The average toxicity reflects the cooperative
population, not the adversary.

\subsection{Implications for Governance Design}

\paragraph{Recommendation 1: Network-aware governance.}
Rather than applying global filtering thresholds, governance systems should
exploit network topology. Agents identified as adversarial could be isolated
\emph{topologically} (reducing their connectivity) rather than
\emph{behaviorally} (tightening acceptance criteria for all). This preserves
interactions among cooperative agents.

\paragraph{Recommendation 2: Adversarial density monitoring.}
The progressive decline pattern appears at ${\sim}35$--$40\%$ adversarial
density. Systems should monitor estimated adversarial fraction and escalate
governance mode when density approaches this threshold---for example, switching
from global filtering to targeted isolation.

\paragraph{Recommendation 3: Faster reputation dynamics.}
The collusion scenario's slow reputation decay (0.95/epoch) means governance
``remembers'' adversarial interactions for ${\sim}20$ epochs (half-life of
$\frac{\ln 2}{-\ln 0.95} \approx 14$ epochs). Faster decay would allow the
system to recover from adversarial episodes more quickly, at the cost of
potentially re-admitting reformed adversaries. The optimal decay rate should
balance recovery speed against adversary re-exploitation.

% ══════════════════════════════════════════════════════════════════════
\section{Connecting to the Liveness--Safety Tradeoff}

The contrasting dynamics in our two scenarios illuminate different points on the
liveness--safety Pareto frontier:

\textbf{Collusion scenario (high adversarial density):} The governance system is
forced to operate in the \emph{safety-dominant} region. With 37.5\% adversarial
agents, maintaining low toxicity requires accepting very few interactions. The
progressive decline is the system gradually sliding along the Pareto frontier
toward maximum safety at the cost of liveness.

\textbf{Network scenario (low adversarial density):} The governance system
operates in the \emph{balanced} region. With 10\% adversarial agents and
network-based isolation, the system can maintain both reasonable toxicity
(0.335) and high liveness (78\% acceptance). The volatility reflects stochastic
exploration of the balanced region rather than systematic movement toward one
extreme.

This suggests the Pareto frontier's shape depends on adversarial density:
\begin{itemize}
  \item At low density: the frontier is nearly flat---liveness and safety are
    cheap simultaneously.
  \item At high density: the frontier is steep---small safety improvements
    require large liveness sacrifices.
\end{itemize}

% ══════════════════════════════════════════════════════════════════════
\section{The Collusion Detection Paradox}

An important negative result: despite collusion detection being explicitly
enabled in the collusion scenario, \textbf{zero collusion pairs were flagged.}
The collusion detection system requires:
\begin{enumerate}
  \item Interaction frequency $z$-score $> 2.0$
  \item Outcome correlation $> 0.7$
  \item Minimum 3 interactions between the pair
\end{enumerate}

Under progressive decline, the interaction count drops rapidly. By epoch~10,
most agent pairs have fewer than 3 interactions, preventing the collusion
detection from accumulating enough data. The governance system's \emph{own
filtering} prevents the collusion detector from gathering the evidence it needs.

This creates a paradox: \textbf{the more effective the behavioral filter, the
less data available for pattern detection.} Collusion detection requires
sustained interaction to build statistical evidence, but the behavioral filter
reduces interactions precisely because it detects problems.

\paragraph{Implication.}
Collusion detection systems need alternative data sources beyond interaction
outcomes---perhaps network structure analysis, communication pattern monitoring,
or controlled ``honeypot'' interactions that generate data even under reduced
throughput.

% ══════════════════════════════════════════════════════════════════════
\section{Related Work}

\paragraph{Social network analysis for fraud detection.}
\citet{akoglu2015graph} survey graph-based anomaly detection, identifying
suspicious subgraphs in social networks. Our network effects results suggest
that dynamic network topology provides \emph{natural} anomaly
containment---the network self-organizes to isolate anomalous agents without
explicit detection.

\paragraph{Adversarial robustness in multi-agent reinforcement learning.}
\citet{gleave2020adversarial} demonstrated that adversarial agents can exploit
MARL policies even when they represent a small fraction of the population. Our
results complement this by showing that the \emph{governance response} to
adversaries, not just the adversaries themselves, can degrade system
performance.

\paragraph{Resilience of complex networks.}
\citet{albert2000error} and \citet{callaway2000network} showed that scale-free
networks are robust to random node failures but vulnerable to targeted attacks.
Our small-world networks show a different resilience pattern: dynamic edge
weights create \emph{temporal} robustness---the network repeatedly recovers
from adversarial periods through edge reconfiguration.

\paragraph{Sybil resistance.}
\citet{douceur2002sybil} and \citet{yu2006sybilguard} showed that Sybil
detection in distributed systems relies on social network structure to bound the
adversary's influence. Our work extends this to settings where the adversary's
influence is governed not just by identity but by interaction quality signals.

% ══════════════════════════════════════════════════════════════════════
\section{Conclusion}

This study reveals two fundamentally different governance dynamics in
multi-agent systems:

\begin{enumerate}
  \item \textbf{Progressive decline} occurs when adversarial density exceeds
    ${\sim}35\%$, causing governance to gradually tighten until the system is
    effectively shut down. The decline follows a characteristic three-phase
    pattern (engagement, transition, attrition) driven by reputation
    contamination and governance momentum.
  \item \textbf{Sustained volatility} occurs when adversarial density is low
    (${\sim}10\%$) and network topology enables local isolation. The system
    maintains throughput with high variance but positive trends in both welfare
    and toxicity.
  \item \textbf{The collusion detection paradox}: effective behavioral filtering
    reduces the data available for pattern detection, creating a fundamental
    tension between reactive and analytical governance approaches.
  \item \textbf{Network-aware governance}---exploiting topology to isolate
    adversaries rather than globally tightening filters---is a promising
    direction for maintaining the liveness--safety balance under adversarial
    pressure.
\end{enumerate}

These findings suggest that the next generation of multi-agent governance
mechanisms should be \emph{topology-aware}, using network structure as both a
detection signal and an intervention lever, rather than relying solely on
interaction-level behavioral filtering.

% ══════════════════════════════════════════════════════════════════════
\bibliography{refs}

% ══════════════════════════════════════════════════════════════════════
\appendix
\section{Data Tables}

\subsection{Collusion Detection---Full Epoch Data}

\begin{table}[ht]
\centering
\small
\caption{Complete epoch-level data for the collusion detection scenario (25 epochs).}
\label{tab:collusion-full}
\begin{tabular}{@{}cccccc@{}}
\toprule
Epoch & Total & Accepted & Rate & Toxicity & Welfare \\
\midrule
 0 & 21 & 16 & 76.2\% & 0.394 & 18.96 \\
 1 & 22 & 14 & 63.6\% & 0.351 & 19.26 \\
 2 & 16 & 10 & 62.5\% & 0.337 & 13.67 \\
 3 & 13 & 12 & 92.3\% & 0.326 & 16.73 \\
 4 & 15 &  5 & 33.3\% & 0.317 &  7.07 \\
 5 & 14 & 10 & 71.4\% & 0.369 & 12.37 \\
 6 & 21 &  8 & 38.1\% & 0.332 & 10.92 \\
 7 & 11 &  4 & 36.4\% & 0.304 &  5.75 \\
 8 & 16 & 10 & 62.5\% & 0.337 & 13.72 \\
 9 &  6 &  3 & 50.0\% & 0.361 &  3.84 \\
10 & 13 &  3 & 23.1\% & 0.356 &  4.05 \\
11 & 15 &  3 & 20.0\% & 0.414 &  3.29 \\
12 &  8 &  1 & 12.5\% & 0.529 &  0.60 \\
13 & 13 &  2 & 15.4\% & 0.418 &  2.16 \\
14 &  8 &  3 & 37.5\% & 0.371 &  3.85 \\
15 &  9 &  3 & 33.3\% & 0.317 &  4.56 \\
16 &  8 &  3 & 37.5\% & 0.438 &  2.99 \\
17 & 13 &  3 & 23.1\% & 0.419 &  3.14 \\
18 &  9 &  4 & 44.4\% & 0.418 &  4.33 \\
19 &  4 &  1 & 25.0\% & 0.262 &  1.66 \\
20 &  6 &  1 & 16.7\% & 0.365 &  1.21 \\
21 & 11 &  3 & 27.3\% & 0.404 &  3.32 \\
22 & 10 &  2 & 20.0\% & 0.315 &  2.85 \\
23 &  9 &  1 & 11.1\% & 0.466 &  0.88 \\
24 &  8 &  2 & 25.0\% & 0.342 &  2.83 \\
\bottomrule
\end{tabular}
\end{table}

\subsection{Network Effects---Full Epoch Data}

\begin{table}[ht]
\centering
\small
\caption{Complete epoch-level data for the network effects scenario (20 epochs).}
\label{tab:network-full}
\begin{tabular}{@{}cccccc@{}}
\toprule
Epoch & Total & Accepted & Rate & Toxicity & Welfare \\
\midrule
 0 & 17 & 15 & 88.2\% & 0.341 & 11.88 \\
 1 & 16 & 12 & 75.0\% & 0.335 &  9.72 \\
 2 &  8 &  6 & 75.0\% & 0.385 &  3.89 \\
 3 & 18 & 15 & 83.3\% & 0.337 & 12.06 \\
 4 & 14 &  9 & 64.3\% & 0.341 &  7.14 \\
 5 & 15 & 12 & 80.0\% & 0.349 &  9.19 \\
 6 & 15 & 12 & 80.0\% & 0.355 &  8.96 \\
 7 & 17 & 12 & 70.6\% & 0.353 &  9.03 \\
 8 & 21 & 19 & 90.5\% & 0.365 & 13.53 \\
 9 & 13 & 10 & 76.9\% & 0.349 &  7.64 \\
10 & 17 & 13 & 76.5\% & 0.351 &  9.86 \\
11 & 13 & 10 & 76.9\% & 0.314 &  8.80 \\
12 & 11 &  6 & 54.5\% & 0.282 &  5.90 \\
13 & 18 & 14 & 77.8\% & 0.344 & 10.93 \\
14 & 15 & 11 & 73.3\% & 0.286 & 10.66 \\
15 & 22 & 18 & 81.8\% & 0.383 & 11.80 \\
16 & 18 & 14 & 77.8\% & 0.310 & 12.50 \\
17 & 19 & 17 & 89.5\% & 0.322 & 14.54 \\
18 &  9 &  8 & 88.9\% & 0.314 &  7.03 \\
19 & 18 & 13 & 72.2\% & 0.278 & 12.94 \\
\bottomrule
\end{tabular}
\end{table}

\end{document}
