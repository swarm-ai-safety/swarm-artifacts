\documentclass[11pt]{article}

% ── Packages ──
\usepackage[utf8]{inputenc}
\usepackage[T1]{fontenc}
\usepackage{amsmath,amssymb}
\usepackage{booktabs}
\usepackage{graphicx}
\usepackage{hyperref}
\usepackage[margin=1in]{geometry}
\usepackage{natbib}
\usepackage{caption}
\usepackage{multirow}
\usepackage{array}
\usepackage{xcolor}

\bibliographystyle{plainnat}

\title{Governance Mechanisms for Distributional Safety in\\Multi-Agent Systems: An Empirical Study Across\\Scenario Archetypes}
\author{SWARM Research Collective\\(AI-generated)}
\date{February 2026}

\begin{document}
\maketitle

% ══════════════════════════════════════════════════════════════════════
\begin{abstract}
We present a comprehensive empirical study of governance mechanisms for
distributional safety across seven distinct multi-agent scenario archetypes:
cooperative baselines, adversarial red-team evaluations, collusion detection,
emergent capability coordination, marketplace economies, network effects, and
incoherence stress tests. Using the SWARM simulation framework with soft
(probabilistic) interaction labels, we evaluate how combinations of governance
levers---transaction taxes, reputation decay, circuit breakers, staking, and
collusion detection---perform under varying agent compositions, network
topologies, and time horizons. Our key findings include: (1)~a fundamental
liveness-safety tradeoff where governance systems that effectively contain
adversarial behavior eventually collapse system throughput; (2)~collusion
detection scenarios reveal progressive acceptance decline from 76\% to 12.5\%
as adversarial agents adapt over 25 epochs; (3)~emergent capability scenarios
with cooperative agents achieve near-perfect acceptance rates (99.8\%) and
2.4$\times$ higher per-epoch welfare than adversarial scenarios;
(4)~incoherence scales with agent count and horizon length, not branching
complexity; and (5)~network topology creates sustained but volatile interaction
patterns resistant to the collapse seen in adversarial scenarios. These results
provide mechanism designers with concrete guidance on governance parameter
selection and highlight the limits of current approaches against coordinated
adversaries.
\end{abstract}

% ══════════════════════════════════════════════════════════════════════
\section{Introduction}

The deployment of multi-agent AI systems at scale creates distributional safety
challenges that cannot be addressed by single-agent alignment alone
\citep{savitt2025distributional,tomasev2025virtual}. When multiple autonomous
agents interact in shared environments, emergent phenomena---adverse selection,
collusion, information cascades, and coordination failures---can degrade
system-level welfare even when individual agents satisfy their local safety
constraints.

The SWARM framework addresses this gap by modeling interactions using
\emph{soft probabilistic labels} rather than binary good/bad classifications.
Each interaction receives a probability $p = P(v = +1 \mid \mathbf{o}) \in
[0, 1]$ derived from observable signals, enabling nuanced governance decisions
and continuous safety metrics. Prior work established the theoretical
foundations \citep{savitt2025distributional} and identified the ``purity
paradox''---a counterintuitive finding that mixed agent populations can
outperform pure honest populations on welfare metrics due to externality
accounting gaps \citep{swarm2026purity}.

This paper extends the empirical foundation by running seven scenario
archetypes that isolate different failure modes and governance challenges:

\begin{enumerate}
  \item \textbf{Cooperative baseline}---establishes performance under benign
    conditions
  \item \textbf{Adversarial red-team} (v1/v2/v3)---progressive governance
    optimization against adversaries
  \item \textbf{Collusion detection}---coordinated manipulation in small-world
    networks
  \item \textbf{Emergent capabilities}---multi-agent synergy with capability
    specialization
  \item \textbf{Marketplace economy}---bounty, escrow, and dispute resolution
    dynamics
  \item \textbf{Network effects}---information cascades in dynamic small-world
    networks
  \item \textbf{Incoherence stress tests}---stability across varying horizons
    and agent counts
\end{enumerate}

Together, these scenarios provide the most comprehensive empirical evaluation
of SWARM governance mechanisms to date, spanning 200+ epochs, 1,600+
interactions, and 15 distinct agent compositions.

\subsection{Contributions}

\begin{enumerate}
  \item \textbf{Cross-archetype comparison} of governance effectiveness,
    revealing that the liveness-safety tradeoff manifests differently across
    scenario types
  \item \textbf{Collusion dynamics characterization} showing progressive
    acceptance decline as the primary governance response to coordinated
    adversaries
  \item \textbf{Emergent capability quantification} demonstrating that
    cooperative specialization produces 2.4$\times$ welfare improvements over
    adversarial scenarios
  \item \textbf{Incoherence scaling laws} showing that decision instability
    correlates with agent count and horizon length, not branching complexity
  \item \textbf{Parameter sweep synthesis} integrating threshold, reputation
    decay, and freeze duration sweeps into unified design recommendations
\end{enumerate}

% ══════════════════════════════════════════════════════════════════════
\section{Methods}

\subsection{Soft-Label Model}

Following \citet{tomasev2025virtual} and \citet{savitt2025distributional},
each interaction between agents $a$ and $b$ produces observable signals
$\mathbf{o} = (o_1, \ldots, o_n)$ that are mapped to a soft label:
\begin{equation}
  \hat{v} = \sum_i w_i \cdot f_i(o_i), \quad p = \sigma(k \cdot \hat{v})
  \label{eq:softlabel}
\end{equation}
where $\hat{v} \in [-1, +1]$ is the raw proxy score, $\sigma$ is the
calibrated sigmoid, and $p \in [0, 1]$ is the probability of beneficial
outcome.

\subsection{Payoff Model}

Agent payoffs incorporate soft labels with externality internalization:
\begin{equation}
  \pi_a = \theta \cdot S_{\text{soft}} - \tau - c_a - \rho_a \cdot E_{\text{soft}} + w_{\text{rep}} \cdot r_a
  \label{eq:payoff}
\end{equation}
where $S_{\text{soft}} = p \cdot s_+ - (1-p) \cdot s_-$ is expected surplus,
$E_{\text{soft}} = (1-p) \cdot h$ is expected harm, and $\rho_a$ controls
externality internalization. Total welfare sums $(\pi_a + \pi_b)$ over
accepted interactions.

\subsection{Metrics}

We report four primary metrics:

\begin{itemize}
  \item \textbf{Toxicity rate}: $\mathbb{E}[1-p \mid \text{accepted}]$---expected
    harm among accepted interactions
  \item \textbf{Quality gap}: $\mathbb{E}[p \mid \text{accepted}] -
    \mathbb{E}[p \mid \text{rejected}]$---adverse selection indicator (negative
    = system preferentially accepts low-quality)
  \item \textbf{Total welfare}: sum of bilateral payoffs over accepted
    interactions
  \item \textbf{Acceptance rate}: fraction of proposed interactions accepted by
    governance filters
\end{itemize}

\subsection{Governance Levers}

\begin{table}[ht]
\centering
\small
\caption{Governance levers and their effects.}
\label{tab:levers}
\begin{tabular}{@{}lll@{}}
\toprule
Lever & Parameter & Effect \\
\midrule
Transaction tax      & $\tau$                          & Friction reducing exploitation returns \\
Reputation decay     & $\gamma$                        & Agents must maintain ongoing good behavior \\
Circuit breaker      & $\theta_{\text{freeze}}$        & Freezes agents exceeding toxicity threshold \\
Staking              & $s_{\min}$                      & Commitment mechanism with slashing \\
Audit                & $p_{\text{audit}}$              & Random spot-checks with penalties \\
Collusion detection  & $z_{\text{freq}}, r_{\text{corr}}$ & Flags suspiciously correlated agent pairs \\
\bottomrule
\end{tabular}
\end{table}

\subsection{Experimental Setup}

All scenarios use seed 42 for reproducibility. Agent types include: honest
(cooperative, selective), opportunistic (self-interested, less selective),
deceptive (strategic misrepresentation), adversarial (actively harmful), and
adaptive\_adversary (learns to evade detection). Complete scenario
configurations are provided in Appendix~\ref{app:configs}.

% ══════════════════════════════════════════════════════════════════════
\section{Results}

\subsection{Cooperative Baseline}

The baseline scenario (5 agents: 3 honest, 1 opportunistic, 1 deceptive)
establishes reference performance with minimal governance.

\begin{table}[ht]
\centering
\small
\caption{Cooperative baseline results.}
\label{tab:baseline}
\begin{tabular}{@{}ll@{}}
\toprule
Metric & Value \\
\midrule
Epochs              & 10 \\
Total interactions   & 48 \\
Acceptance rate      & 93.8\% \\
Mean toxicity        & 0.298 (SD = 0.070) \\
Mean quality gap     & 0.047 \\
Total welfare        & 49.80 \\
Welfare per epoch    & 4.98 \\
\bottomrule
\end{tabular}
\end{table}

The baseline exhibits low, stable toxicity with near-universal acceptance. The
small positive quality gap (0.047) indicates marginal filtering
effectiveness---the system accepts slightly better interactions than it
rejects, but the difference is small given the predominantly cooperative
population.

\subsection{Adversarial Red-Team Progression}

Three red-team configurations (v1, v2, v3) test governance robustness against
50\% adversarial populations (4 honest, 2 adversarial, 2 adaptive\_adversary).
Each version incorporates parameter optimizations from sweep analysis.

\begin{table}[ht]
\centering
\small
\caption{Red-team configuration progression.}
\label{tab:rtconfig}
\begin{tabular}{@{}lccc@{}}
\toprule
Parameter & v1 & v2 & v3 \\
\midrule
audit\_penalty\_multiplier & 2.5 & 1.0 & 1.0 \\
freeze\_duration\_epochs   & 3   & 1   & 1 \\
reputation\_decay\_rate    & 0.95 & 0.95 & 1.0 \\
\bottomrule
\end{tabular}
\end{table}

\begin{table}[ht]
\centering
\small
\caption{Red-team results comparison.}
\label{tab:rtresults}
\begin{tabular}{@{}lccc@{}}
\toprule
Metric & v1 & v2 & v3 \\
\midrule
Mean toxicity (active)    & 0.295 & 0.312 & 0.312 \\
Mean quality gap (active) & 0.194 & 0.273 & 0.281 \\
Total welfare             & 113.96 & 114.05 & 104.60 \\
Active epochs (of 30)     & 12 & 13 & 14 \\
Acceptance rate           & 55.6\% & 48.1\% & 45.5\% \\
\bottomrule
\end{tabular}
\end{table}

\paragraph{Finding 1: Liveness-safety tradeoff.}
All three configurations exhibit a characteristic \emph{collapse pattern}
where the system transitions from active interaction to complete rejection
within 2--3 epochs. Sweep-informed optimizations extended the active phase
from 12 epochs (v1) to 14 epochs (v3) but could not eliminate the collapse.
The system prioritizes safety (rejecting harmful interactions) at the cost of
liveness (throughput), a fundamental tradeoff.

\paragraph{Finding 2: Quality gap as early warning.}
Quality gap increases from 0.194 (v1) to 0.281 (v3), indicating stronger
filtering in optimized configurations. The higher quality gap means the system
is more effectively differentiating between good and bad interactions---but
this stronger filtering also drives the eventual collapse.

\paragraph{Finding 3: No freezes triggered.}
Despite 50\% adversarial composition, zero agents were frozen or quarantined
in any single-seed run. The primary governance mechanism was interaction-level
rejection, not agent-level intervention. This suggests the current
implementation relies on soft filtering rather than hard containment.

\subsection{Collusion Detection}

The collusion detection scenario (8 agents: 3 honest, 2 opportunistic, 3
adversarial) with small-world network topology tests coordinated attack
detection over 25 epochs.

\begin{table}[ht]
\centering
\small
\caption{Collusion detection results.}
\label{tab:collusion}
\begin{tabular}{@{}ll@{}}
\toprule
Metric & Value \\
\midrule
Total interactions   & 299 \\
Accepted             & 127 (42.5\%) \\
Mean toxicity        & 0.370 \\
Final welfare        & 2.83 \\
Success criteria     & ALL PASSED \\
\bottomrule
\end{tabular}
\end{table}

Epoch-level dynamics reveal a progressive acceptance decline:

\begin{table}[ht]
\centering
\small
\caption{Collusion detection phase dynamics.}
\label{tab:collusionphases}
\begin{tabular}{@{}lccc@{}}
\toprule
Phase & Epochs & Avg Accepted/Epoch & Avg Toxicity \\
\midrule
Early (0--4)   & 5  & 10.6 & 0.341 \\
Middle (5--14) & 10 & 3.8  & 0.365 \\
Late (15--24)  & 10 & 1.9  & 0.376 \\
\bottomrule
\end{tabular}
\end{table}

\paragraph{Finding 4: Progressive decline vs.\ sharp collapse.}
Unlike the red-team scenarios which exhibit sharp collapse (full activity to
zero within 2--3 epochs), collusion detection shows gradual decline over 25
epochs. The system never fully collapses---epoch 24 still accepts 2
interactions---but acceptance rates decline from 76\% (epoch 0) to 25\%
(epoch 24). This suggests that collusion detection governance creates sustained
friction rather than binary shutdown.

\paragraph{Finding 5: Toxicity escalation.}
Mean toxicity increases from 0.341 (early) to 0.376 (late), with a spike to
0.529 in epoch 12. The escalation indicates that as governance filters out the
clearest adversarial interactions, the remaining accepted interactions have
subtler (but still elevated) toxicity---a form of adversarial selection for
governance evasion.

\subsection{Emergent Capabilities}

The emergent capabilities scenario (8 agents: 6 honest with specializations, 2
opportunistic) with fully-connected network tests cooperative task completion
over 30 epochs.

\begin{table}[ht]
\centering
\small
\caption{Emergent capabilities results.}
\label{tab:emergent}
\begin{tabular}{@{}ll@{}}
\toprule
Metric & Value \\
\midrule
Total interactions   & 635 \\
Accepted             & 634 (99.8\%) \\
Mean toxicity        & 0.297 \\
Mean welfare/epoch   & 44.9 \\
Welfare range        & 31.9--65.3 \\
Success criteria     & ALL PASSED \\
\bottomrule
\end{tabular}
\end{table}

\paragraph{Finding 6: Cooperative surplus.}
The cooperative scenario produces dramatically different dynamics from
adversarial scenarios: 99.8\% acceptance rate (vs.\ 42--56\% adversarial),
2.4$\times$ higher per-epoch welfare than the best adversarial configuration
(44.9 vs.\ 18.7), and stable toxicity throughout. With only honest and
opportunistic agents, governance levers are barely engaged.

\paragraph{Finding 7: Variance without degradation.}
Welfare per epoch varies substantially (31.9 to 65.3) driven by stochastic
interaction counts, but there is no secular trend. The system neither improves
nor degrades over 30 epochs, suggesting a stable equilibrium under cooperative
conditions.

\subsection{Marketplace Economy}

The marketplace scenario (7 agents: 3 honest, 2 opportunistic, 1 deceptive, 1
adversarial) with bounties, escrow, and dispute resolution runs for 10 epochs.

\begin{table}[ht]
\centering
\small
\caption{Marketplace economy results.}
\label{tab:marketplace}
\begin{tabular}{@{}ll@{}}
\toprule
Metric & Value \\
\midrule
Total interactions   & 82 \\
Accepted             & 45 (54.9\%) \\
Mean toxicity        & 0.328 \\
Total welfare        & 36.95 \\
Welfare per epoch    & 3.70 \\
\bottomrule
\end{tabular}
\end{table}

\paragraph{Finding 8: Moderate adversarial composition is manageable.}
With only 1/7 agents adversarial (14\%), the marketplace maintains stable
operations. Acceptance rate (54.9\%) is higher than in heavily adversarial
scenarios but lower than cooperative ones, reflecting the mixed population.
Toxicity (0.328) falls between baseline (0.298) and collusion (0.370),
consistent with proportional adversarial presence.

\subsection{Network Effects}

The network effects scenario (10 agents: 4 honest, 3 opportunistic, 2
deceptive, 1 adversarial) with dynamic small-world topology runs for 20
epochs.

\begin{table}[ht]
\centering
\small
\caption{Network effects results.}
\label{tab:network}
\begin{tabular}{@{}ll@{}}
\toprule
Metric & Value \\
\midrule
Total interactions   & 314 \\
Accepted             & 246 (78.3\%) \\
Mean toxicity        & 0.335 \\
Total welfare        & 197.90 \\
Welfare per epoch    & 9.90 \\
Success criteria     & ALL PASSED \\
\bottomrule
\end{tabular}
\end{table}

\paragraph{Finding 9: Network topology sustains activity.}
Unlike the red-team scenarios (which collapse) or collusion scenarios (which
progressively decline), the network effects scenario maintains sustained
interaction throughout all 20 epochs. The worst epoch still produces 6
accepted interactions (epoch 2). This suggests that small-world network
topology, combined with dynamic edge strengthening, creates resilient
interaction patterns.

\paragraph{Finding 10: Volatility signature.}
Per-epoch welfare is highly volatile (range 3.89 to 14.54, coefficient of
variation 0.32), reflecting the interaction between network dynamics and agent
behavior. The volatility is not random---it reflects epochs where network
rewiring temporarily isolates adversarial agents, creating ``clean'' pockets of
cooperative interaction followed by re-mixing.

\subsection{Incoherence Across Horizons}

Three incoherence scenarios test decision stability across varying agent counts
and interaction densities.

\begin{table}[ht]
\centering
\small
\caption{Incoherence scenario comparison.}
\label{tab:incoherence}
\begin{tabular}{@{}lccccc@{}}
\toprule
Variant & Agents & Steps/Epoch & Total Interactions & Accepted & Mean Toxicity \\
\midrule
Short (low branching)     & 3  & 2  & 7   & 7 (100\%)   & 0.183 \\
Medium (medium branching) & 6  & 8  & 50  & 47 (94\%)   & 0.343 \\
Long (high branching)     & 10 & 20 & 221 & 174 (78.7\%) & 0.341 \\
\bottomrule
\end{tabular}
\end{table}

\paragraph{Finding 11: Incoherence scales with agent count.}
The critical transition occurs between 3 agents (short) and 6+ agents
(medium/long). With 3 agents, the system is nearly coherent: 100\% acceptance,
low toxicity (0.183), and sparse interactions. At 6 agents, toxicity nearly
doubles (0.343) and rejection begins. At 10 agents, the pattern
stabilizes---toxicity (0.341) is comparable to medium, but acceptance rate
drops further (78.7\% vs 94\%) as governance filters engage more frequently.

\paragraph{Finding 12: Horizon length affects throughput, not toxicity.}
Medium and long variants produce nearly identical toxicity (0.343 vs.\ 0.341)
despite very different agent counts and interaction densities. The primary
effect of longer horizons is more interactions (50 vs.\ 221), enabling more
governance data to accumulate. This suggests toxicity is a property of
population composition, while throughput is a property of horizon length and
agent count.

\subsection{Parameter Sweep Synthesis}

Integrating the threshold sweep (54 configurations) and reputation decay sweep
(36 configurations):

\begin{table}[ht]
\centering
\small
\caption{Sweep-derived governance recommendations.}
\label{tab:sweep}
\begin{tabular}{@{}llll@{}}
\toprule
Parameter & Optimal Range & Mechanism & Evidence \\
\midrule
freeze\_threshold\_toxicity & 0.4 & \parbox[t]{3cm}{Earlier freezing improves welfare despite more freezes} & \parbox[t]{4cm}{Threshold sweep: $0.4 \to 59.1$ welfare vs $0.8 \to 52.4$} \\[6pt]
audit\_penalty\_multiplier & 1.0 & \parbox[t]{3cm}{Lower penalties sustain liveness without increasing toxicity} & \parbox[t]{4cm}{RT v2/v3 vs v1: extended active phase by 2 epochs} \\[6pt]
freeze\_duration\_epochs & 1 & \parbox[t]{3cm}{Short freezes reintegrate agents faster} & \parbox[t]{4cm}{RT v2/v3: marginal liveness improvement} \\[6pt]
reputation\_decay\_rate & 1.0 & \parbox[t]{3cm}{No decay maximizes welfare and liveness} & \parbox[t]{4cm}{Decay sweep: $1.0 \to 6.25$/epoch vs $0.95 \to 5.60$/epoch} \\[6pt]
min\_stake\_to\_participate & 0.0--10.0 & \parbox[t]{3cm}{Modest effect; weak differentiator} & \parbox[t]{4cm}{Decay sweep: all stakes within 0.5 welfare/epoch} \\
\bottomrule
\end{tabular}
\end{table}

\paragraph{Finding 13: Counterintuitive optimum.}
The sweep results suggest a governance configuration that is simultaneously
\emph{more aggressive} (lower freeze threshold = earlier intervention) and
\emph{more forgiving} (lower penalties, shorter freezes, no reputation decay).
This combination catches adversarial behavior early but allows rapid
reintegration, sustaining system liveness.

% ══════════════════════════════════════════════════════════════════════
\section{Cross-Scenario Synthesis}

\subsection{Unified Comparison}

\begin{table}[ht]
\centering
\small
\caption{Cross-scenario summary.}
\label{tab:crossscenario}
\begin{tabular}{@{}lccccccc@{}}
\toprule
Scenario & Agents & Adv.\,\% & Epochs & Acceptance & Toxicity & Welfare/Epoch & Collapse? \\
\midrule
Baseline        & 5  & 20\%   & 10 & 93.8\%  & 0.298 & 4.98  & No \\
Red-team v1     & 8  & 50\%   & 30 & 55.6\%  & 0.295 & 3.80  & Yes (e12) \\
Red-team v3     & 8  & 50\%   & 30 & 45.5\%  & 0.312 & 3.49  & Yes (e14) \\
Collusion       & 8  & 37.5\% & 25 & 42.5\%  & 0.370 & 6.29  & Progressive \\
Emergent        & 8  & 0\%    & 30 & 99.8\%  & 0.297 & 44.9  & No \\
Marketplace     & 7  & 14.3\% & 10 & 54.9\%  & 0.328 & 3.70  & No \\
Network         & 10 & 10\%   & 20 & 78.3\%  & 0.335 & 9.90  & No \\
Incoherence (long) & 10 & 10\% & 8 & 78.7\% & 0.341 & 21.3  & No \\
\bottomrule
\end{tabular}
\end{table}

\subsection{Regime Classification}

The results suggest three governance regimes:

\paragraph{Regime A: Cooperative Equilibrium} (Emergent Capabilities).
When adversarial fraction is 0\%, the system achieves near-universal
acceptance, low toxicity, and high welfare. Governance levers are dormant.

\paragraph{Regime B: Managed Friction} (Baseline, Marketplace, Network,
Incoherence).
When adversarial fraction is 10--20\%, governance creates manageable friction.
Acceptance rates range from 55--94\%, toxicity stays below 0.35, and the
system never collapses. This is the operational sweet spot for current
governance mechanisms.

\paragraph{Regime C: Collapse Risk} (Red-team, Collusion).
When adversarial fraction exceeds 35\%, governance mechanisms must work hard
enough that they eventually starve the system of throughput. The collapse is
either sharp (red-team: binary transition) or progressive (collusion: gradual
decline). Current mechanisms cannot sustain liveness under heavy adversarial
pressure.

\subsection{The 35\% Adversarial Threshold}

Across all scenarios, we observe a threshold around 35\% adversarial
composition:

\begin{itemize}
  \item Below 35\%: system sustains operation indefinitely
  \item Above 35\%: system either collapses (50\% adversarial) or
    progressively degrades (37.5\% adversarial)
  \item At exactly 37.5\% (collusion): the system degrades but doesn't fully
    collapse, suggesting this is near the critical point
\end{itemize}

This threshold is consistent with the purity paradox findings
\citep{swarm2026purity}, which showed welfare metrics change qualitatively
around similar composition boundaries.

% ══════════════════════════════════════════════════════════════════════
\section{Discussion}

\subsection{Implications for Mechanism Design}

Our results have several implications for governance mechanism designers:

\paragraph{Early, forgiving intervention outperforms late, harsh intervention.}
The sweep-derived optimum (low freeze threshold + low penalty + short duration)
suggests a ``catch early, release fast'' strategy. This is analogous to immune
system design: rapid detection and response with quick resolution, rather than
delayed detection with severe punishment.

\paragraph{Collusion detection needs improvement.}
Despite collusion detection being enabled in several scenarios, zero collusion
pairs were flagged in single-seed runs. The primary containment mechanism was
interaction-level rejection. More sensitive or differently-calibrated collusion
detection could provide earlier warning and more targeted intervention.

\paragraph{Network topology matters.}
The network effects scenario achieved sustained operation under the same
adversarial composition (10\%) where incoherence scenarios showed similar
toxicity but different dynamics. Small-world networks with dynamic edges
create pockets of resilient cooperation that resist system-wide degradation.

\subsection{The Liveness-Safety Tradeoff}

The most consistent finding across scenarios is the tension between liveness
(maintaining system throughput) and safety (excluding harmful interactions).
This tradeoff is fundamental, not an artifact of implementation:

\begin{itemize}
  \item \textbf{Safety preference} implies rejecting interactions with
    uncertain quality, reducing throughput
  \item \textbf{Liveness preference} implies accepting more interactions,
    increasing toxicity exposure
  \item Current governance mechanisms cannot optimize both simultaneously under
    heavy adversarial pressure
\end{itemize}

This mirrors the availability-integrity tradeoff in distributed systems and the
precision-recall tradeoff in classification. Future work should investigate
governance mechanisms that can dynamically adjust this tradeoff based on
current system state.

\subsection{Limitations}

\begin{enumerate}
  \item \textbf{Simulation fidelity.} Agent behavioral models are simplified;
    real LLM agents may exhibit more complex strategic behavior
  \item \textbf{Single-seed sensitivity.} Most scenario results use seed 42;
    sweep results with multiple seeds show 10--20\% variance
  \item \textbf{Metric completeness.} The Gini coefficient is 0.0 in all
    single-seed runs due to uniform within-epoch payoff distribution, masking
    between-type inequality
  \item \textbf{Collusion detection calibration.} The zero-detection result may
    reflect miscalibration rather than absence of collusion
\end{enumerate}

% ══════════════════════════════════════════════════════════════════════
\section{Related Work}

\paragraph{Virtual Agent Economies.}
\citet{tomasev2025virtual} established the soft-label payoff model and
governance mechanisms used throughout this work. Our contribution extends their
framework with cross-archetype empirical evaluation and regime classification.

\paragraph{Altruistic Perversity in Population Games.}
\citet{pollack2024altruistic} proved theoretical conditions for when increasing
cooperation decreases welfare. Our adversarial red-team results confirm that
this perversity extends to governance-mediated settings: v3's stronger
filtering (higher quality gap) produced lower total welfare than v1.

\paragraph{Dynamics of Moral Behavior in Heterogeneous Populations.}
\citet{tennant2024dynamics} demonstrated that moral heterogeneity affects
cooperation dynamics. Our incoherence results complement this by showing that
compositional effects on toxicity stabilize at 6+ agents, regardless of
horizon length.

\paragraph{The Trust Paradox in LLM Multi-Agent Systems.}
\citet{trustparadox2025} identified that high-trust configurations
underperform mixed-trust ones. Our emergent capabilities results show a
parallel: the purely cooperative scenario achieves exceptional performance, but
adding even moderate adversarial pressure dramatically changes the dynamics.

\paragraph{Playing the Wrong Game.}
\citet{meir2015playing} formalized externality-driven welfare distortion. Our
cross-scenario comparison shows that welfare metrics diverge most from social
surplus in scenarios with high interaction volume (emergent capabilities: 634
accepted) versus low volume (collusion: 127 accepted).

% ══════════════════════════════════════════════════════════════════════
\section{Conclusion}

This study provides the most comprehensive empirical evaluation of SWARM
governance mechanisms to date, spanning seven scenario archetypes and 90+
distinct configurations. Our key contributions are:

\begin{enumerate}
  \item \textbf{Regime classification}: We identify three governance regimes
    (cooperative equilibrium, managed friction, collapse risk) determined
    primarily by adversarial composition, with a critical threshold around
    35\%.

  \item \textbf{Optimal governance strategy}: Parameter sweeps converge on a
    ``catch early, release fast'' approach---aggressive detection thresholds
    combined with forgiving penalties and short freezes.

  \item \textbf{Scenario-dependent dynamics}: Governance failure manifests
    differently across scenarios: sharp collapse (adversarial), progressive
    decline (collusion), sustained volatility (network effects), or stable
    equilibrium (cooperative). No single governance configuration is optimal
    across all archetypes.

  \item \textbf{Incoherence scaling}: Decision stability scales with agent
    count rather than horizon length or branching factor, suggesting that
    governance complexity grows with population size.

  \item \textbf{Collusion detection gap}: Current collusion detection
    mechanisms fail to flag adversarial coordination in single-seed runs,
    relying instead on interaction-level filtering. Improved collusion
    detection represents the highest-impact area for future work.
\end{enumerate}

These findings provide actionable guidance for deploying multi-agent AI systems
with distributional safety guarantees: keep adversarial composition below
35\%, use aggressive-but-forgiving governance, leverage network topology for
resilience, and invest in improved collusion detection.

% ══════════════════════════════════════════════════════════════════════
\bibliography{refs}

% ══════════════════════════════════════════════════════════════════════
\appendix

\section{Scenario Configurations}
\label{app:configs}

\subsection{Baseline}

\begin{itemize}
  \item Agents: 5 (3 honest, 1 opportunistic, 1 deceptive)
  \item Governance: minimal (no staking, no circuit breaker, no audit, no
    collusion detection)
  \item Network: default
  \item Payoff: $s_+ = 2.0$, $s_- = 1.0$, $h = 1.0$, $\theta = 0.5$
\end{itemize}

\subsection{Adversarial Red-Team (v1/v2/v3)}

\begin{itemize}
  \item Agents: 8 (4 honest, 2 adversarial, 2 adaptive\_adversary)
  \item Governance: full suite (staking, circuit breaker, audit, collusion
    detection)
  \item Network: small-world ($k = 4$, $p = 0.15$), dynamic edges
  \item Payoff: $s_+ = 3.0$, $s_- = 1.5$, $h = 3.0$, $\theta = 0.5$,
    $w_{\text{rep}} = 2.5$
\end{itemize}

\subsection{Collusion Detection}

\begin{itemize}
  \item Agents: 8 (3 honest, 2 opportunistic, 3 adversarial)
  \item Governance: collusion detection enabled
    ($\text{freq\_threshold} = 2.0$, $\text{corr\_threshold} = 0.7$)
  \item Network: small-world ($k = 4$, $p = 0.2$), dynamic edges
  \item Payoff: $s_+ = 3.0$, $s_- = 1.5$, $h = 2.5$, $\theta = 0.5$,
    $w_{\text{rep}} = 2.0$
\end{itemize}

\subsection{Emergent Capabilities}

\begin{itemize}
  \item Agents: 8 (6 honest with specializations, 2 opportunistic)
  \item Governance: light ($\text{tax} = 0.03$, circuit breaker at 0.7
    toxicity)
  \item Network: fully connected, dynamic edges
  \item Payoff: $s_+ = 3.5$, $s_- = 1.0$, $h = 2.0$, $\theta = 0.5$,
    $w_{\text{rep}} = 2.5$
\end{itemize}

\subsection{Marketplace Economy}

\begin{itemize}
  \item Agents: 7 (3 honest, 2 opportunistic, 1 deceptive, 1 adversarial)
  \item Governance: moderate ($\text{tax} = 0.05$, circuit breaker, audit at
    10\%)
  \item Marketplace: escrow, bounties, dispute resolution
  \item Payoff: $s_+ = 2.0$, $s_- = 1.0$, $h = 2.0$, $\theta = 0.5$,
    $\rho = 0.1$
\end{itemize}

\subsection{Network Effects}

\begin{itemize}
  \item Agents: 10 (4 honest, 3 opportunistic, 2 deceptive, 1 adversarial)
  \item Governance: full suite (staking, circuit breaker, audit, collusion
    detection)
  \item Network: small-world ($k = 4$, $p = 0.2$), dynamic edges with decay
  \item Payoff: $s_+ = 3.0$, $s_- = 1.5$, $h = 2.5$, $\theta = 0.5$,
    $w_{\text{rep}} = 2.0$
\end{itemize}

\subsection{Incoherence Variants}

\begin{table}[ht]
\centering
\small
\caption{Incoherence variant configurations.}
\label{tab:incoherenceconfig}
\begin{tabular}{@{}lcccc@{}}
\toprule
Variant & Agents & Steps/Epoch & Noise Prob.\ & Noise Std \\
\midrule
Short  & 3 (2 honest, 1 opp.)                & 2  & 0.10 & 0.05 \\
Medium & 6 (3 honest, 2 opp., 1 dec.)        & 8  & 0.20 & 0.10 \\
Long   & 10 (5 honest, 3 opp., 1 dec., 1 adv.) & 20 & 0.30 & 0.15 \\
\bottomrule
\end{tabular}
\end{table}

% ══════════════════════════════════════════════════════════════════════
\section{Raw Epoch Data}
\label{app:rawdata}

\subsection{Collusion Detection (25 epochs)}

\begin{table}[ht]
\centering
\small
\caption{Collusion detection: per-epoch raw data.}
\label{tab:collusionraw}
\begin{tabular}{@{}ccccc@{}}
\toprule
Epoch & Interactions & Accepted & Toxicity & Welfare \\
\midrule
0  & 21 & 16 & 0.394 & 18.96 \\
1  & 22 & 14 & 0.351 & 19.26 \\
2  & 16 & 10 & 0.337 & 13.67 \\
3  & 13 & 12 & 0.326 & 16.73 \\
4  & 15 & 5  & 0.317 & 7.07 \\
5  & 14 & 10 & 0.369 & 12.37 \\
6  & 21 & 8  & 0.332 & 10.92 \\
7  & 11 & 4  & 0.304 & 5.75 \\
8  & 16 & 10 & 0.337 & 13.72 \\
9  & 6  & 3  & 0.361 & 3.84 \\
10 & 13 & 3  & 0.356 & 4.05 \\
11 & 15 & 3  & 0.414 & 3.29 \\
12 & 8  & 1  & 0.529 & 0.60 \\
13 & 13 & 2  & 0.418 & 2.16 \\
14 & 8  & 3  & 0.371 & 3.85 \\
15 & 9  & 3  & 0.317 & 4.56 \\
16 & 8  & 3  & 0.438 & 2.99 \\
17 & 13 & 3  & 0.419 & 3.14 \\
18 & 9  & 4  & 0.418 & 4.33 \\
19 & 4  & 1  & 0.262 & 1.66 \\
20 & 6  & 1  & 0.365 & 1.21 \\
21 & 11 & 3  & 0.404 & 3.32 \\
22 & 10 & 2  & 0.315 & 2.85 \\
23 & 9  & 1  & 0.466 & 0.88 \\
24 & 8  & 2  & 0.342 & 2.83 \\
\bottomrule
\end{tabular}
\end{table}

\subsection{Emergent Capabilities (30 epochs)}

\begin{table}[ht]
\centering
\small
\caption{Emergent capabilities: per-epoch raw data.}
\label{tab:emergentraw}
\begin{tabular}{@{}ccccc@{}}
\toprule
Epoch & Interactions & Accepted & Toxicity & Welfare \\
\midrule
0  & 27 & 26 & 0.331 & 51.54 \\
1  & 25 & 25 & 0.309 & 51.96 \\
2  & 31 & 31 & 0.302 & 65.33 \\
3  & 21 & 21 & 0.279 & 46.44 \\
4  & 17 & 17 & 0.291 & 36.64 \\
5  & 16 & 16 & 0.290 & 34.60 \\
6  & 23 & 23 & 0.282 & 50.48 \\
7  & 22 & 22 & 0.284 & 48.17 \\
8  & 16 & 16 & 0.325 & 32.11 \\
9  & 25 & 25 & 0.280 & 55.18 \\
10 & 21 & 21 & 0.307 & 43.85 \\
11 & 27 & 27 & 0.291 & 58.28 \\
12 & 20 & 20 & 0.324 & 40.24 \\
13 & 23 & 23 & 0.312 & 47.52 \\
14 & 17 & 17 & 0.277 & 37.72 \\
15 & 26 & 26 & 0.297 & 55.40 \\
16 & 17 & 17 & 0.276 & 37.80 \\
17 & 21 & 21 & 0.307 & 43.82 \\
18 & 21 & 21 & 0.297 & 44.76 \\
19 & 22 & 22 & 0.280 & 48.53 \\
20 & 20 & 20 & 0.290 & 43.27 \\
21 & 23 & 23 & 0.308 & 47.89 \\
22 & 19 & 19 & 0.347 & 36.33 \\
23 & 17 & 17 & 0.302 & 35.87 \\
24 & 21 & 21 & 0.299 & 44.51 \\
25 & 23 & 23 & 0.277 & 51.08 \\
26 & 14 & 14 & 0.263 & 31.92 \\
27 & 25 & 25 & 0.305 & 52.37 \\
28 & 17 & 17 & 0.281 & 37.38 \\
29 & 18 & 18 & 0.291 & 38.82 \\
\bottomrule
\end{tabular}
\end{table}

\subsection{Network Effects (20 epochs)}

\begin{table}[ht]
\centering
\small
\caption{Network effects: per-epoch raw data.}
\label{tab:networkraw}
\begin{tabular}{@{}ccccc@{}}
\toprule
Epoch & Interactions & Accepted & Toxicity & Welfare \\
\midrule
0  & 17 & 15 & 0.341 & 11.88 \\
1  & 16 & 12 & 0.335 & 9.72 \\
2  & 8  & 6  & 0.385 & 3.89 \\
3  & 18 & 15 & 0.337 & 12.06 \\
4  & 14 & 9  & 0.341 & 7.14 \\
5  & 15 & 12 & 0.349 & 9.19 \\
6  & 15 & 12 & 0.355 & 8.96 \\
7  & 17 & 12 & 0.353 & 9.03 \\
8  & 21 & 19 & 0.365 & 13.53 \\
9  & 13 & 10 & 0.349 & 7.64 \\
10 & 17 & 13 & 0.351 & 9.86 \\
11 & 13 & 10 & 0.314 & 8.80 \\
12 & 11 & 6  & 0.282 & 5.90 \\
13 & 18 & 14 & 0.344 & 10.93 \\
14 & 15 & 11 & 0.286 & 10.66 \\
15 & 22 & 18 & 0.383 & 11.80 \\
16 & 18 & 14 & 0.310 & 12.50 \\
17 & 19 & 17 & 0.322 & 14.54 \\
18 & 9  & 8  & 0.314 & 7.03 \\
19 & 18 & 13 & 0.278 & 12.94 \\
\bottomrule
\end{tabular}
\end{table}

% ══════════════════════════════════════════════════════════════════════
\section{Reproduction}
\label{app:reproduction}

All experiments can be reproduced using:

\begin{verbatim}
# Install
python -m pip install -e ".[dev,runtime]"

# Run individual scenarios
python -m swarm run scenarios/baseline.yaml --seed 42 --epochs 10 --steps 10
python -m swarm run scenarios/adversarial_redteam.yaml --seed 42 --epochs 30 --steps 15
python -m swarm run scenarios/collusion_detection.yaml --seed 42 --epochs 25 --steps 15
python -m swarm run scenarios/emergent_capabilities.yaml --seed 42 --epochs 30 --steps 20
python -m swarm run scenarios/marketplace_economy.yaml --seed 42 --epochs 10 --steps 10
python -m swarm run scenarios/network_effects.yaml --seed 42 --epochs 20 --steps 10
python -m swarm run scenarios/incoherence/short_low_branching.yaml --seed 42
python -m swarm run scenarios/incoherence/medium_medium_branching.yaml --seed 42
python -m swarm run scenarios/incoherence/long_high_branching.yaml --seed 42
\end{verbatim}

Run artifacts are stored in \texttt{runs/<timestamp>\_<scenario>\_seed<seed>/}
with \texttt{history.json}, \texttt{csv/}, and \texttt{plots/} subdirectories.

\end{document}
