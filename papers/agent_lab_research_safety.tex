\documentclass[11pt,a4paper]{article}
\usepackage[utf8]{inputenc}
\usepackage[T1]{fontenc}
\usepackage{amsmath,amssymb}
\usepackage{booktabs}
\usepackage{graphicx}
\usepackage{hyperref}
\usepackage[margin=1in]{geometry}
\usepackage{caption}
\usepackage{subcaption}
\usepackage{xcolor}

\title{Governance of Autonomous Research Pipelines:\\
A Distributional Safety Study of AgentLaboratory under SWARM}

\author{Raeli Savitt}
\date{February 2026}

\begin{document}
\maketitle

\begin{abstract}
We study the distributional safety profile of autonomous research pipelines
governed by SWARM, using AgentLaboratory---a system that orchestrates six
specialized LLM agents through literature review, experimentation, code
execution, and paper writing---as the target domain.  We sweep three
governance levers (transaction tax rate, circuit breaker, collusion detection)
across 16 configurations with 10 pre-registered seeds each (160 total runs).
Our main finding is that transaction taxation significantly reduces welfare
and honest agent payoffs: a 10\% tax decreases welfare by 8.1\% relative to
the no-tax baseline ($p = 0.0007$, Cohen's $d = 0.80$, survives
Bonferroni correction at $\alpha/32$).  Neither circuit breakers nor collusion
detection show significant main effects in this all-honest population.
Toxicity rates remain stable around 26\% across all configurations,
and no adverse selection is observed (quality gap = 0).  These results
suggest that in cooperative research pipelines, governance overhead from
transaction taxes imposes a measurable welfare cost without a corresponding
safety benefit, while binary safety mechanisms (circuit breakers, collusion
detection) are inert when the agent population is benign.
\end{abstract}

\section{Introduction}

As AI systems become capable of conducting autonomous research
\cite{schmidgall2025agentlaboratory}, the question of how to govern these
workflows becomes pressing.  AgentLaboratory orchestrates six specialized
LLM agents---PhD Student, Postdoc, Professor, ML Engineer, Software Engineer,
and a three-member Reviewer panel---through a four-phase research pipeline:
literature review, experimentation, interpretation, and paper writing.

SWARM provides a distributional safety framework that models agent
interactions probabilistically, computing soft labels $p \in [0,1]$ via
proxy observables rather than binary good/bad classifications.  This paper
bridges the two systems, asking: \emph{What is the distributional safety
profile of an autonomous research pipeline under varying governance regimes?}

We focus on three governance levers:
\begin{enumerate}
    \item \textbf{Transaction tax rate} ($\tau \in \{0\%, 3\%, 6\%, 10\%\}$):
    A per-interaction tax that funds the governance commons.
    \item \textbf{Circuit breaker} (on/off): Freezes agents whose toxicity
    exceeds a threshold.
    \item \textbf{Collusion detection} (on/off): Monitors reviewer panels
    for correlated scoring patterns.
\end{enumerate}

\section{Scenario Design}

\subsection{Agent Population}

\begin{table}[h]
\centering
\caption{Agent population mapped from AgentLaboratory roles to SWARM agent IDs.}
\label{tab:agents}
\begin{tabular}{llcc}
\toprule
AgentLab Role & SWARM Agent ID & Type & Count \\
\midrule
PhDStudentAgent & \texttt{agent\_lab\_phd} & honest & 1 \\
PostdocAgent & \texttt{agent\_lab\_postdoc} & honest & 1 \\
ProfessorAgent & \texttt{agent\_lab\_professor} & honest & 1 \\
MLEngineerAgent & \texttt{agent\_lab\_mle} & honest & 1 \\
SWEngineerAgent & \texttt{agent\_lab\_swe} & honest & 1 \\
ReviewersAgent & \texttt{agent\_lab\_reviewer} & honest & 3 \\
\midrule
\multicolumn{3}{l}{\textbf{Total}} & \textbf{8} \\
\bottomrule
\end{tabular}
\end{table}

All agents are honest type, reflecting the cooperative baseline of an
autonomous research pipeline where no adversarial agents are injected.

\subsection{Governance Configuration}

\begin{table}[h]
\centering
\caption{Fixed governance parameters (not swept).}
\label{tab:governance}
\begin{tabular}{lc}
\toprule
Parameter & Value \\
\midrule
Transaction tax split & 0.5 \\
Reputation decay rate & 0.95 \\
Freeze threshold (toxicity) & 0.6 \\
Freeze threshold (violations) & 5 \\
Freeze duration (epochs) & 1 \\
Collusion frequency threshold & 2.0 \\
Collusion correlation threshold & 0.7 \\
Audit probability & 0.15 \\
Audit penalty multiplier & 2.0 \\
Bandwidth cap & 20 \\
\bottomrule
\end{tabular}
\end{table}

\subsection{Payoff Configuration}

The payoff function uses $s^+ = 3.0$, $s^- = 1.5$, $h = 2.5$ (moderate
harm from research quality degradation), $\theta = 0.5$,
$\rho_a = 0.3$, $\rho_b = 0.2$, $w_{\text{rep}} = 1.0$.

\subsection{Simulation Parameters}

Each run simulates 4 epochs (mapping to AgentLab's four research phases)
with 20 steps per epoch.  Seeds 42--51 are pre-registered (10 seeds per
configuration).

\section{Sweep Design}

\begin{table}[h]
\centering
\caption{Swept parameters and their values.}
\label{tab:sweep}
\begin{tabular}{lcc}
\toprule
Parameter & Values & Levels \\
\midrule
Transaction tax rate ($\tau$) & 0.0, 0.03, 0.06, 0.10 & 4 \\
Circuit breaker enabled & False, True & 2 \\
Collusion detection enabled & False, True & 2 \\
\midrule
\multicolumn{2}{l}{\textbf{Total configurations}} & \textbf{16} \\
\multicolumn{2}{l}{\textbf{Seeds per configuration}} & 10 \\
\multicolumn{2}{l}{\textbf{Total runs}} & \textbf{160} \\
\bottomrule
\end{tabular}
\end{table}

\section{Results}

\subsection{Welfare}

\begin{table}[h]
\centering
\caption{Welfare by transaction tax rate (aggregated across CB and CD settings).}
\label{tab:welfare}
\begin{tabular}{ccccc}
\toprule
Tax Rate & Mean & SD & Median & $n$ \\
\midrule
0\% & 113.0 & 9.3 & 112.2 & 40 \\
3\% & 107.8 & 11.1 & 107.1 & 40 \\
6\% & 107.2 & 13.0 & 110.7 & 40 \\
10\% & 103.8 & 13.4 & 103.2 & 40 \\
\bottomrule
\end{tabular}
\end{table}

Welfare decreases monotonically with tax rate.  The 0\% vs.\ 10\% comparison
is the only pairwise contrast that survives Bonferroni correction
($p = 0.0007$, adjusted $p = 0.021$, Cohen's $d = 0.80$, medium effect).
The 0\% vs.\ 3\% ($p = 0.026$, $d = 0.51$) and 0\% vs.\ 6\% ($p = 0.024$,
$d = 0.51$) comparisons reach nominal significance but do not survive
multiple comparisons correction.

\begin{figure}[h]
\centering
\includegraphics[width=0.8\textwidth]{figures/agent_lab_research_safety/welfare_by_tax.png}
\caption{Welfare vs.\ transaction tax rate with 95\% confidence intervals.}
\label{fig:welfare_tax}
\end{figure}

\subsection{Toxicity}

Toxicity rates are remarkably stable across all configurations, ranging
from 25.9\% to 26.8\%.  No pairwise comparison on toxicity survives
Bonferroni correction.  The largest nominal effect is tax 0\% vs.\ 10\%
($p = 0.052$, $d = -0.44$), approaching but not reaching significance.

\begin{figure}[h]
\centering
\includegraphics[width=0.8\textwidth]{figures/agent_lab_research_safety/toxicity_by_tax.png}
\caption{Toxicity rate vs.\ transaction tax rate with 95\% confidence intervals.}
\label{fig:toxicity_tax}
\end{figure}

\subsection{Circuit Breaker and Collusion Detection}

Neither circuit breaker ($p = 0.86$, $d = 0.03$) nor collusion detection
($p = 0.15$, $d = 0.23$) shows a significant main effect on welfare.
This is expected: in an all-honest population, these mechanisms have no
adversarial behavior to detect or contain.

\begin{figure}[h]
\centering
\includegraphics[width=0.8\textwidth]{figures/agent_lab_research_safety/welfare_by_mechanism.png}
\caption{Welfare by governance mechanism combination (CB = circuit breaker,
CD = collusion detection).}
\label{fig:welfare_mechanism}
\end{figure}

\subsection{Honest Agent Payoff}

Honest agent payoffs track welfare exactly (all agents are honest, so
per-agent payoff $\approx$ welfare / 8).  The 0\% vs.\ 10\% tax comparison
again survives Bonferroni correction ($p = 0.0007$, $d = 0.80$).

\begin{figure}[h]
\centering
\includegraphics[width=0.8\textwidth]{figures/agent_lab_research_safety/honest_payoff_by_config.png}
\caption{Honest agent payoff vs.\ tax rate, split by circuit breaker status.}
\label{fig:payoff}
\end{figure}

\subsection{Quality Gap and Adverse Selection}

Quality gap is identically zero across all configurations.  With only
honest agents, there is no mechanism for adverse selection: all interactions
are accepted with comparable $p$ values, producing no quality differential
between accepted and rejected interactions.

\subsection{Heatmap: Tax Rate $\times$ Circuit Breaker}

\begin{figure}[h]
\centering
\includegraphics[width=0.7\textwidth]{figures/agent_lab_research_safety/heatmap_tax_cb.png}
\caption{Mean welfare heatmap across tax rate and circuit breaker settings.}
\label{fig:heatmap}
\end{figure}

The heatmap reveals that the worst-performing configuration is
$\tau = 10\%$ with CB on and CD on (welfare = 95.0), while the best is
$\tau = 0\%$ with CB off and CD on (welfare = 118.7).

\section{Statistical Methodology}

\subsection{Pre-Registration}

Seeds 42--51 were declared before running the sweep.  All 32 hypothesis
tests are enumerated in the P-hacking audit table (available in
\texttt{summary.json}).

\subsection{Multiple Comparisons}

With 32 hypothesis tests, we apply both Bonferroni correction
($\alpha_{\text{adj}} = 0.05/32 = 0.00156$) and Holm-Bonferroni
step-down correction.  Both methods yield the same 2 surviving tests.

\subsection{Effect Sizes}

We report Cohen's $d$ (pooled standard deviation) for all comparisons.
The surviving findings have $d = 0.80$ (medium/large boundary).

\subsection{Normality}

Shapiro-Wilk tests confirm normality for all per-tax-rate welfare
distributions ($p > 0.05$ in all cases), validating the use of
Welch's $t$-test.  Mann-Whitney $U$ tests provide non-parametric
robustness checks with concordant results.

\section{Discussion}

\subsection{Tax as Governance Overhead}

The central finding---that transaction taxes reduce welfare without
improving safety---highlights a fundamental tension in governance design
for cooperative systems.  In hostile environments with adversarial agents,
taxes may deter exploitation by making low-quality interactions unprofitable.
But in cooperative research pipelines, they act as pure friction, reducing
the surplus available to honest agents.

This result is consistent with mechanism design theory: taxes are
second-best instruments that achieve their effect through deadweight loss.
When the first-best outcome (no adverse behavior) is already achieved
by agent selection, adding governance overhead is strictly welfare-reducing.

\subsection{Inert Safety Mechanisms}

Circuit breakers and collusion detection show no effect because the
all-honest population never triggers their activation conditions.
The toxicity freeze threshold (0.6) is far above the observed toxicity
rates ($\sim$0.26), and the collusion correlation threshold (0.7) exceeds
any natural reviewer agreement patterns.

This raises an important design question: \emph{Should governance
mechanisms have zero cost when not activated?}  Our results suggest they
do: neither mechanism adds measurable overhead in the inactive state.

\subsection{Limitations}

\begin{enumerate}
    \item \textbf{All-honest population}: The most important limitation.
    Future work should introduce adversarial agents (opportunistic
    reviewers, a deceptive MLE agent) to test whether governance levers
    become protective.
    \item \textbf{Simulated interactions}: The bridge maps AgentLab
    roles to SWARM agents but does not run actual LLM inference.
    Real research quality variance may differ.
    \item \textbf{Fixed payoff parameters}: The $h = 2.5$ harm parameter
    and acceptance thresholds are fixed; sweeping these may reveal
    regime changes.
\end{enumerate}

\section{Reproducibility}

All results can be reproduced from:

\begin{verbatim}
python runs/20260213-204503_agent_lab_research_safety_study/run_sweep.py
python runs/20260213-204503_agent_lab_research_safety_study/analyze.py
python runs/20260213-204503_agent_lab_research_safety_study/generate_plots.py
\end{verbatim}

Scenario: \texttt{scenarios/agent\_lab\_research\_safety.yaml} \\
Seeds: 42--51 (pre-registered) \\
Commit: see \texttt{git log} for the study run tag.

\section{Conclusion}

In a cooperative autonomous research pipeline, governance through
transaction taxation imposes a measurable welfare cost ($d = 0.80$
at 10\% tax) without safety benefit.  Binary safety mechanisms
(circuit breakers, collusion detection) are inert when no adversarial
agents are present, but importantly incur no overhead either.  These
baseline results establish the reference distribution against which
future adversarial studies should be compared.

\bibliographystyle{plain}
\begin{thebibliography}{1}
\bibitem{schmidgall2025agentlaboratory}
S.~Schmidgall, Y.~Harris, et al.
\newblock AgentLaboratory: Using LLM Agents as Research Assistants.
\newblock \emph{arXiv preprint arXiv:2501.04227}, 2025.
\end{thebibliography}

\end{document}
