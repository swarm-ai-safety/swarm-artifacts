\documentclass[11pt]{article}
\usepackage[margin=1in]{geometry}
\usepackage{booktabs}
\usepackage{graphicx}
\usepackage{hyperref}
\usepackage{amsmath}
\usepackage{amssymb}
\usepackage{caption}
\usepackage{array}
\usepackage{longtable}
\usepackage{float}

\title{Governance Parameter Effects on Recursive Collusion Dynamics\\in Multi-Agent Systems}
\author{Raeli Savitt}
\date{February 2026}

\begin{document}
\maketitle

\begin{abstract}
We investigate how transaction taxes and circuit breakers affect ecosystem outcomes in a multi-agent scenario designed to test implicit collusion through recursive reasoning. Using 80 simulation runs (8 governance configurations $\times$ 10 pre-registered seeds) with 12 agents (9 RLM agents at reasoning depths 1, 3, and 5, plus 3 honest baseline agents), we find that transaction tax rate has a statistically significant monotonic negative effect on welfare (0\% vs 15\%: Welch's $t = 4.19$, $p = 0.0002$, Cohen's $d = 1.33$) and a corresponding positive effect on toxicity ($t = -7.74$, $p < 0.0001$, $d = -2.45$). Both findings survive Bonferroni correction across all 12 hypotheses tested. Circuit breakers show no detectable effect on any metric (welfare: $p = 0.93$, $d = -0.018$; toxicity: $p = 0.85$, $d = 0.043$). Per-agent analysis reveals that honest agents earn significantly more than RLM agents (592.98 vs 214.89, $p = 0.00002$, $d = 1.88$), indicating that recursive reasoning does not confer a payoff advantage under active collusion detection governance. All normality assumptions are validated by Shapiro-Wilk tests, and 10 of 12 hypotheses survive Benjamini-Hochberg correction at FDR $= 0.05$.
\end{abstract}

\section{Introduction}

Recursive reasoning in multi-agent systems creates a theoretical pathway to implicit collusion: agents that model other agents' reasoning (level-$k$ thinking) may converge on cooperative strategies without explicit communication. This raises a governance question---can standard mechanisms (taxes, circuit breakers, audits) prevent coordination that emerges from shared reasoning rather than direct signaling?

This study tests two governance levers:

\begin{enumerate}
    \item \textbf{Transaction tax rate} (0\%, 5\%, 10\%, 15\%): Taxes reduce surplus from all interactions, potentially discouraging low-margin collusive trades.
    \item \textbf{Circuit breakers} (on/off): Freeze agents exceeding toxicity or violation thresholds, potentially catching colluding agents.
\end{enumerate}

We hold constant a governance baseline of collusion detection (pair-wise frequency and correlation monitoring), random auditing (15\% probability), and bandwidth caps, testing whether taxes and circuit breakers provide marginal protection beyond this baseline.

\section{Experimental Setup}

\subsection{Scenario}

\begin{table}[H]
\centering
\caption{Scenario configuration.}
\label{tab:scenario}
\small
\begin{tabular}{@{}ll@{}}
\toprule
Property & Value \\
\midrule
Scenario ID & \texttt{rlm\_recursive\_collusion} \\
Motif & \texttt{recursive\_collusion} \\
Epochs & 5 \\
Steps/Epoch & 15 \\
Total agents & 12 \\
\bottomrule
\end{tabular}
\end{table}

\subsection{Agent Composition}

\begin{table}[H]
\centering
\caption{Agent groups. RLM agents use level-$k$ recursive reasoning at varying depths.}
\label{tab:agents}
\small
\begin{tabular}{@{}llcccc@{}}
\toprule
Group & Type & Count & Recursion Depth & Planning Horizon & Memory \\
\midrule
\texttt{rlm\_shallow} & RLM & 3 & 1 & 3 & 100 \\
\texttt{rlm\_mid} & RLM & 3 & 3 & 5 & 100 \\
\texttt{rlm\_deep} & RLM & 3 & 5 & 7 & 100 \\
\texttt{honest} & Honest & 3 & --- & --- & --- \\
\bottomrule
\end{tabular}
\end{table}

\subsection{Governance Baseline (Held Constant)}

\begin{table}[H]
\centering
\caption{Fixed governance parameters across all sweep configurations.}
\label{tab:governance}
\small
\begin{tabular}{@{}ll@{}}
\toprule
Parameter & Value \\
\midrule
Collusion detection & Enabled \\
Collusion frequency threshold & 2.0 \\
Collusion correlation threshold & 0.7 \\
Collusion score threshold & 0.5 \\
Collusion penalty multiplier & 1.5 \\
Audit probability & 0.15 \\
Audit penalty multiplier & 2.0 \\
Bandwidth cap & 15 \\
\bottomrule
\end{tabular}
\end{table}

\subsection{Sweep Configuration}

\begin{table}[H]
\centering
\caption{Parameter sweep design: 4 tax levels $\times$ 2 circuit breaker states $\times$ 10 seeds = 80 runs.}
\label{tab:sweep}
\small
\begin{tabular}{@{}ll@{}}
\toprule
Parameter & Values \\
\midrule
\texttt{transaction\_tax\_rate} & 0.0, 0.05, 0.10, 0.15 \\
\texttt{circuit\_breaker\_enabled} & True, False \\
Seeds per configuration & 10 (pre-registered: 43--122) \\
\textbf{Total runs} & \textbf{80} \\
\bottomrule
\end{tabular}
\end{table}

\subsection{Payoff Parameters}

\begin{table}[H]
\centering
\caption{Payoff engine configuration.}
\label{tab:payoff}
\small
\begin{tabular}{@{}lcl@{}}
\toprule
Parameter & Value & Description \\
\midrule
$s_+$ & 2.0 & Surplus if beneficial \\
$s_-$ & 1.0 & Cost if harmful \\
$h$ & 2.0 & Externality harm \\
$\theta$ & 0.5 & Surplus split \\
$\rho_a$, $\rho_b$ & 0.0 & No externality internalization \\
$w_{\text{rep}}$ & 1.0 & Reputation weight \\
\bottomrule
\end{tabular}
\end{table}

\subsection{Metrics}

\begin{itemize}
    \item \textbf{Toxicity}: $\mathbb{E}[1-p \mid \text{accepted}]$---expected harm from accepted interactions.
    \item \textbf{Welfare per epoch}: total payoff across all agents per epoch.
    \item \textbf{Quality gap}: $\mathbb{E}[p \mid \text{accepted}] - \mathbb{E}[p \mid \text{rejected}]$.
    \item \textbf{Honest agent payoff}: mean payoff for honest-type agents.
    \item \textbf{Acceptance rate}: fraction of interactions accepted by governance.
\end{itemize}

\section{Results}

\subsection{Cross-Configuration Summary}

\begin{table}[H]
\centering
\caption{Per-configuration means across 10 seeds each.}
\label{tab:results}
\small
\begin{tabular}{@{}cccccccc@{}}
\toprule
Tax & CB & Toxicity & Welf./Ep. & Qual.\ Gap & Honest \$ & Accept \% & Avg Rep \\
\midrule
0\% & Off & 0.336 & 102.59 & 0.012 & 65.34 & 88.2\% & 7.088 \\
0\% & On & 0.336 & 105.33 & 0.012 & 68.67 & 89.0\% & 7.271 \\
5\% & Off & 0.340 & 102.52 & 0.021 & 68.10 & 90.4\% & 2.633 \\
5\% & On & 0.339 & 103.71 & 0.020 & 71.03 & 91.0\% & 2.793 \\
10\% & Off & 0.343 & 97.72 & 0.024 & 68.24 & 91.6\% & 1.416 \\
10\% & On & 0.341 & 98.61 & 0.014 & 68.14 & 91.1\% & 1.611 \\
15\% & Off & 0.346 & 95.50 & 0.015 & 70.48 & 92.8\% & 0.341 \\
15\% & On & 0.347 & 91.27 & 0.027 & 62.86 & 91.9\% & 0.053 \\
\bottomrule
\end{tabular}
\end{table}

\subsection{Tax Rate Effect}

Table~\ref{tab:tax} reports tax-level aggregates (pooling over circuit breaker setting, $n = 20$ per level).

\begin{table}[H]
\centering
\caption{Tax rate effect aggregated over circuit breaker (mean $\pm$ SD).}
\label{tab:tax}
\small
\begin{tabular}{@{}cccc@{}}
\toprule
Tax Rate & Welfare/Epoch & Toxicity & Honest Payoff \\
\midrule
0\% & $103.96 \pm 9.62$ & $0.336 \pm 0.004$ & $67.01 \pm 16.36$ \\
5\% & $103.11 \pm 5.85$ & $0.339 \pm 0.005$ & $69.57 \pm 9.20$ \\
10\% & $98.16 \pm 5.33$ & $0.342 \pm 0.004$ & $68.19 \pm 9.12$ \\
15\% & $93.39 \pm 5.89$ & $0.347 \pm 0.005$ & $66.67 \pm 10.53$ \\
\bottomrule
\end{tabular}
\end{table}

Welfare declines \textbf{10.2\%} from 0\% to 15\% tax (103.96 to 93.39). The relationship is monotonically decreasing across all four levels.

\subsection{Statistical Tests}

\subsubsection{All Pairwise Tax Comparisons}

We enumerate 12 hypotheses: 6 pairwise tax comparisons $\times$ 2 metrics (welfare, toxicity). The Bonferroni-corrected threshold is $\alpha = 0.05/12 = 0.004167$.

\begin{table}[H]
\centering
\caption{P-hacking audit: all 12 hypotheses sorted by $p$-value. Corrections: Bonferroni ($\alpha/12$) and Benjamini-Hochberg (FDR $= 0.05$).}
\label{tab:phack}
\small
\begin{tabular}{@{}clccccccc@{}}
\toprule
\# & Comparison & Metric & Welch's $t$ & $p$ & $d$ & MW-$U$ $p$ & Bonf. & BH \\
\midrule
1 & 0\% vs 15\% & Toxicity & $-7.74$ & $< 10^{-6}$ & $-2.45$ & $< 10^{-6}$ & \checkmark & \checkmark \\
2 & 5\% vs 15\% & Welfare & $5.24$ & $6 \times 10^{-6}$ & $1.66$ & $3 \times 10^{-5}$ & \checkmark & \checkmark \\
3 & 5\% vs 15\% & Toxicity & $-5.01$ & $1 \times 10^{-5}$ & $-1.59$ & $5 \times 10^{-5}$ & \checkmark & \checkmark \\
4 & 0\% vs 10\% & Toxicity & $-4.39$ & $9 \times 10^{-5}$ & $-1.39$ & $4 \times 10^{-4}$ & \checkmark & \checkmark \\
5 & 0\% vs 15\% & Welfare & $4.19$ & $2 \times 10^{-4}$ & $1.33$ & $8 \times 10^{-4}$ & \checkmark & \checkmark \\
6 & 10\% vs 15\% & Toxicity & $-3.76$ & $6 \times 10^{-4}$ & $-1.19$ & $8 \times 10^{-4}$ & \checkmark & \checkmark \\
7 & 5\% vs 10\% & Welfare & $2.80$ & $0.008$ & $0.88$ & $0.008$ & & \checkmark \\
8 & 10\% vs 15\% & Welfare & $2.69$ & $0.011$ & $0.85$ & $0.009$ & & \checkmark \\
9 & 0\% vs 5\% & Toxicity & $-2.50$ & $0.017$ & $-0.79$ & $0.024$ & & \checkmark \\
10 & 0\% vs 10\% & Welfare & $2.36$ & $0.025$ & $0.75$ & $0.057$ & & \checkmark \\
11 & 5\% vs 10\% & Toxicity & $-1.63$ & $0.112$ & $-0.52$ & $0.114$ & & \\
12 & 0\% vs 5\% & Welfare & $0.34$ & $0.738$ & $0.11$ & $0.925$ & & \\
\bottomrule
\end{tabular}

\vspace{4pt}
\footnotesize 6/12 survive Bonferroni; 10/12 survive Benjamini-Hochberg (FDR $= 0.05$).
\end{table}

\subsubsection{Circuit Breaker Null Effect}

\begin{table}[H]
\centering
\caption{Circuit breaker effect ($n = 40$ per group).}
\label{tab:cb}
\small
\begin{tabular}{@{}lccc@{}}
\toprule
Metric & $t$-statistic & $p$-value & Cohen's $d$ \\
\midrule
Welfare & $-0.082$ & $0.935$ & $-0.018$ \\
Toxicity & $0.192$ & $0.849$ & $0.043$ \\
\bottomrule
\end{tabular}
\end{table}

Circuit breakers have no detectable effect on any outcome metric.

\subsubsection{Per-Agent Group Comparison}

\begin{table}[H]
\centering
\caption{Per-agent group payoffs across 10 seeds.}
\label{tab:agents-results}
\small
\begin{tabular}{@{}lccc@{}}
\toprule
Group & $N$ & Mean Payoff & SD \\
\midrule
Honest & 30 & 592.98 & 406.50 \\
RLM (all depths) & 90 & 214.89 & 7.55 \\
\bottomrule
\end{tabular}

\vspace{4pt}
\footnotesize Honest vs RLM: $t = 5.09$, $p = 0.00002$, $d = 1.88$ (Bonferroni-significant).
\end{table}

Honest agents earn \textbf{2.76$\times$} more than RLM agents on average, though with substantially higher variance (SD $= 406.50$ vs $7.55$).

\subsubsection{Normality Validation}

\begin{table}[H]
\centering
\caption{Shapiro-Wilk normality tests ($n = 20$ per group). All $p > 0.40$.}
\label{tab:normality}
\small
\begin{tabular}{@{}ccccc@{}}
\toprule
Tax & Welfare $W$ & Welfare $p$ & Toxicity $W$ & Toxicity $p$ \\
\midrule
0\% & 0.952 & 0.402 & 0.969 & 0.731 \\
5\% & 0.959 & 0.528 & 0.980 & 0.928 \\
10\% & 0.963 & 0.607 & 0.972 & 0.797 \\
15\% & 0.957 & 0.482 & 0.974 & 0.840 \\
\bottomrule
\end{tabular}
\end{table}

\subsection{Figures}

\begin{figure}[H]
\centering
\includegraphics[width=0.85\textwidth]{../../runs/20260210-213833_collusion_governance/plots/welfare_vs_tax.png}
\caption{Welfare per epoch decreases monotonically with transaction tax rate. Error bars show 95\% CI across 20 runs per point. The 0\% vs 15\% comparison survives Bonferroni correction ($p = 0.0002$, $d = 1.33$).}
\label{fig:welfare}
\end{figure}

\begin{figure}[H]
\centering
\includegraphics[width=0.85\textwidth]{../../runs/20260210-213833_collusion_governance/plots/toxicity_vs_tax.png}
\caption{Toxicity increases with tax rate ($p < 0.0001$, $d = 2.45$).}
\label{fig:toxicity}
\end{figure}

\begin{figure}[H]
\centering
\includegraphics[width=0.85\textwidth]{../../runs/20260210-213833_collusion_governance/plots/welfare_toxicity_tradeoff.png}
\caption{Welfare-toxicity tradeoff by configuration. Circuit breaker settings overlap within each tax level, visually confirming the null CB effect.}
\label{fig:tradeoff}
\end{figure}

\begin{figure}[H]
\centering
\includegraphics[width=0.85\textwidth]{../../runs/20260210-213833_collusion_governance/plots/quality_gap_vs_tax.png}
\caption{Quality gap remains positive across all configurations.}
\label{fig:qgap}
\end{figure}

\begin{figure}[H]
\centering
\includegraphics[width=0.85\textwidth]{../../runs/20260210-213833_collusion_governance/plots/honest_payoff_vs_tax.png}
\caption{Honest agent payoff vs tax rate.}
\label{fig:honest}
\end{figure}

\begin{figure}[H]
\centering
\includegraphics[width=0.85\textwidth]{../../runs/20260210-213833_collusion_governance/plots/circuit_breaker_null.png}
\caption{Box plots confirming the circuit breaker null effect for both welfare ($p = 0.93$) and toxicity ($p = 0.85$).}
\label{fig:cbnull}
\end{figure}

\section{Discussion}

\subsection{Tax as Pure Deadweight Loss}

Transaction taxes reduce welfare monotonically without compensating benefits. Toxicity \emph{increases} with tax rate, meaning taxes make the ecosystem both poorer and less safe. The mechanism is straightforward: taxes reduce $S_{\text{soft}} = p \cdot s_+ - (1-p) \cdot s_-$ uniformly, disproportionately penalizing high-quality interactions that generate the most surplus.

\subsection{Circuit Breaker Redundancy}

The null circuit breaker effect ($d < 0.05$ on both metrics) indicates complete functional redundancy with the existing governance stack. The collusion detection system (pair-wise frequency and correlation monitoring at thresholds freq $= 2.0$, corr $= 0.7$) and random auditing (15\%) are sufficient to prevent the behaviors circuit breakers would catch. This suggests circuit breakers may only become relevant at higher adversarial fractions or without collusion detection.

\subsection{Recursive Reasoning Does Not Confer Advantage}

The 2.76$\times$ honest-over-RLM payoff gap ($d = 1.88$) is striking: deeper recursive reasoning does not translate to higher payoffs under collusion detection governance. The RLM agents' narrow payoff variance (SD $= 7.55$) compared to honest agents (SD $= 406.50$) suggests they converge on similar low-risk strategies regardless of reasoning depth, while honest agents benefit from higher-variance but higher-expected-value interactions.

\subsection{Reputation Erosion Under Tax}

Average reputation drops sharply with tax rate: from 7.18 at 0\% tax to 0.20 at 15\%. This secondary effect may compound the welfare loss---as reputation erodes, the reputation-weighted payoff component ($w_{\text{rep}} = 1.0$) contributes less, further reducing returns to participation.

\section{Limitations}

\begin{enumerate}
    \item \textbf{Short time horizon} (5 epochs): Longer runs may reveal tax adaptation or reputation recovery dynamics.
    \item \textbf{No adaptive adversaries}: RLM agents follow fixed recursive strategies rather than adapting to governance pressure.
    \item \textbf{Collusion detection always on}: We do not test tax/CB interaction \emph{without} collusion detection.
    \item \textbf{Single scenario}: Results may not generalize to scenarios with explicit adversarial agents.
    \item \textbf{RLM group homogeneity}: Per-agent analysis grouped all RLM depths together ($n = 90$). Depth-stratified analysis would require more seeds.
\end{enumerate}

\section{Reproducibility}

All results can be reproduced from committed artifacts:

\begin{verbatim}
python -c "
import sys; sys.path.insert(0, '.')
from pathlib import Path
from swarm.analysis import SweepConfig, SweepParameter, SweepRunner
from swarm.scenarios import load_scenario

scenario = load_scenario(Path('scenarios/rlm_recursive_collusion.yaml'))
scenario.orchestrator_config.n_epochs = 5

config = SweepConfig(
    base_scenario=scenario,
    parameters=[
        SweepParameter(name='governance.transaction_tax_rate',
                       values=[0.0, 0.05, 0.10, 0.15]),
        SweepParameter(name='governance.circuit_breaker_enabled',
                       values=[False, True]),
    ],
    runs_per_config=10, seed_base=42,
)
runner = SweepRunner(config)
runner.run()
runner.to_csv(Path('sweep_results.csv'))
"
\end{verbatim}

Raw data: \texttt{runs/20260210-213833\_collusion\_governance/sweep\_results.csv}

\begin{thebibliography}{3}

\bibitem{savitt2026a}
Savitt, R. (2026).
Distributional {AGI} safety: Governance trade-offs in multi-agent systems under adversarial pressure.
\emph{SWARM Technical Report}.

\bibitem{savitt2026b}
Savitt, R. (2026).
Transaction taxes reduce welfare monotonically while circuit breakers show null effect.
\emph{SWARM Technical Report}.

\bibitem{swarm}
SWARM Framework. \url{https://github.com/swarm-ai-safety/swarm}

\end{thebibliography}

\end{document}
