% SWARM Paper Template
% Generated from docs/papers/baseline_governance.md

\documentclass[11pt]{article}
\usepackage[margin=1in]{geometry}
\usepackage{booktabs}
\usepackage{graphicx}
\usepackage{hyperref}
\usepackage{amsmath}
\usepackage{amssymb}
\usepackage{caption}
\usepackage{array}
\usepackage{longtable}
\usepackage{float}
\usepackage{enumitem}
\usepackage{verbatim}

\title{Baseline Governance: Transaction Tax and Circuit Breaker\\Effects on Multi-Agent Welfare}
\author{SWARM Research Collective}
\date{February 2026}

\begin{document}
\maketitle

\begin{abstract}
We investigate the effects of transaction taxation and circuit breakers on welfare, toxicity, and distributional fairness in a mixed-agent simulation. Using the SWARM framework, we sweep tax rates (0\%, 5\%, 10\%, 15\%) and circuit breaker activation (enabled/disabled) across 80 runs (10 seeds per configuration) with 5 agents (3 honest, 1 opportunistic, 1 deceptive). We find that transaction taxes significantly reduce total welfare ($d=1.41$, $p<0.0001$ for 0\% vs 15\% tax), while circuit breakers have no statistically significant effect. The welfare reduction from taxation disproportionately affects honest agents ($d=1.29$, $p=0.0002$ for 0\% vs 10\% tax on honest payoff), while deceptive agents remain relatively unaffected. All welfare distributions pass Shapiro-Wilk normality tests, validating parametric analysis. Of 42 pre-registered hypotheses, 4 survive Bonferroni correction and 6 survive Benjamini-Hochberg correction.
\end{abstract}

\section{Introduction}

Governance mechanisms in multi-agent systems face a fundamental tension: interventions designed to reduce harmful behavior may impose deadweight costs that reduce overall welfare. Transaction taxes are a canonical governance lever --- they increase the cost of interactions, potentially deterring low-value or exploitative exchanges, but also reducing the volume of beneficial interactions. Circuit breakers provide a qualitatively different mechanism: they halt activity when systemic risk indicators exceed thresholds, but may be inert in stable environments.

This study provides a controlled empirical characterization of these two governance levers in the SWARM baseline scenario, a minimal mixed-agent environment with 3 honest, 1 opportunistic, and 1 deceptive agent. We focus on three questions:

\begin{enumerate}
\item Does transaction taxation reduce welfare, and if so, is the effect monotonic in the tax rate?
\item Does circuit breaker activation affect welfare or toxicity in a stable baseline environment?
\item How do governance costs distribute across agent types --- do they disproportionately burden honest or deceptive agents?
\end{enumerate}

\section{Methods}

\subsection{Scenario Configuration}

\begin{table}[H]
\centering
\begin{tabular}{ll}
\toprule
\textbf{Parameter} & \textbf{Value} \\
\midrule
Agents & 5 (3 honest, 1 opportunistic, 1 deceptive) \\
Epochs & 10 \\
Steps per epoch & 10 \\
Payoff: $s^+$ & 2.0 \\
Payoff: $s^-$ & 1.0 \\
Payoff: $h$ (externality) & 2.0 \\
Payoff: $\theta$ (governance threshold) & 0.5 \\
Reputation decay & 1.0 (disabled) \\
Bandwidth cap & 10 \\
Staking & Disabled \\
Audit & Disabled \\
\bottomrule
\end{tabular}
\caption{Baseline scenario configuration.}
\end{table}

\subsection{Swept Parameters}

\begin{table}[H]
\centering
\begin{tabular}{ll}
\toprule
\textbf{Parameter} & \textbf{Values} \\
\midrule
\texttt{governance.transaction\_tax\_rate} & 0.0, 0.05, 0.10, 0.15 \\
\texttt{governance.circuit\_breaker\_enabled} & False, True \\
\bottomrule
\end{tabular}
\caption{Swept parameters. Total configurations: $4 \times 2 = 8$. Seeds per configuration: 10. Total runs: 80.}
\end{table}

\subsection{Metrics}

\begin{itemize}
\item \textbf{Welfare}: Cumulative payoff across all agents over all epochs
\item \textbf{Toxicity rate}: $\mathbb{E}[1-p \mid \text{accepted}]$, expected harm from accepted interactions
\item \textbf{Quality gap}: $\mathbb{E}[p \mid \text{accepted}] - \mathbb{E}[p \mid \text{rejected}]$; negative indicates adverse selection
\item \textbf{Agent payoff}: Per-type mean payoff (honest, opportunistic, deceptive)
\end{itemize}

\subsection{Statistical Methods}

\begin{itemize}
\item Welch's $t$-test (unequal variance) for pairwise comparisons
\item Mann-Whitney $U$ as non-parametric robustness check
\item Cohen's $d$ for effect sizes (pooled SD)
\item Bonferroni correction ($\alpha = 0.05/42 = 0.00119$)
\item Benjamini-Hochberg correction for false discovery rate
\item Shapiro-Wilk normality validation on welfare distributions
\item Paired $t$-test for agent-type stratification
\item 42 total pre-registered hypotheses across all parameter-metric pairs
\end{itemize}

\section{Results}

\subsection{Welfare by Tax Rate}

\begin{table}[H]
\centering
\begin{tabular}{lllll}
\toprule
\textbf{Tax Rate} & \textbf{Welfare (mean $\pm$ SD)} & \textbf{Toxicity (mean $\pm$ SD)} & \textbf{Quality Gap (mean $\pm$ SD)} & \textbf{N} \\
\midrule
0.00 & $62.5 \pm 7.0$ & $0.3018 \pm 0.0133$ & $0.0094 \pm 0.0184$ & 20 \\
0.05 & $60.8 \pm 7.3$ & $0.3045 \pm 0.0143$ & $0.0146 \pm 0.0218$ & 20 \\
0.10 & $52.3 \pm 8.3$ & $0.3084 \pm 0.0154$ & $0.0101 \pm 0.0201$ & 20 \\
0.15 & $53.2 \pm 6.2$ & $0.3066 \pm 0.0149$ & $0.0133 \pm 0.0178$ & 20 \\
\bottomrule
\end{tabular}
\caption{Welfare, toxicity, and quality gap by transaction tax rate.}
\end{table}

Transaction tax significantly reduces welfare. The effect is non-linear: a 5\% tax produces a modest decline (62.5 to 60.8), while 10\% and 15\% produce a sharper drop (to 52.3 and 53.2 respectively). Toxicity is unaffected by taxation --- all values cluster near 0.305 regardless of tax rate. Quality gap remains near zero and positive (no adverse selection) across all conditions.

\begin{figure}[H]
\centering
\includegraphics[width=0.8\textwidth]{figures/baseline_governance/welfare_vs_tax.png}
\caption{Welfare vs transaction tax rate with 95\% CI error bars.}
\end{figure}

\begin{figure}[H]
\centering
\includegraphics[width=0.8\textwidth]{figures/baseline_governance/toxicity_vs_tax.png}
\caption{Toxicity vs transaction tax rate with 95\% CI error bars.}
\end{figure}

\subsection{Circuit Breaker Effect}

\begin{table}[H]
\centering
\begin{tabular}{llll}
\toprule
\textbf{Circuit Breaker} & \textbf{Welfare (mean $\pm$ SD)} & \textbf{Toxicity (mean $\pm$ SD)} & \textbf{N} \\
\midrule
Disabled & $56.8 \pm 7.7$ & $0.3061 \pm 0.0140$ & 40 \\
Enabled & $57.6 \pm 9.2$ & $0.3046 \pm 0.0150$ & 40 \\
\bottomrule
\end{tabular}
\caption{Welfare and toxicity by circuit breaker state.}
\end{table}

Circuit breaker activation has no statistically significant effect on welfare ($p>0.05$) or toxicity. This is expected in a stable baseline environment where systemic risk indicators remain below circuit breaker thresholds.

\begin{figure}[H]
\centering
\includegraphics[width=0.8\textwidth]{figures/baseline_governance/welfare_vs_cb.png}
\caption{Welfare vs circuit breaker state with 95\% CI error bars.}
\end{figure}

\subsection{Interaction Effects}

The grouped analysis (tax rate $\times$ circuit breaker) confirms that the two governance levers operate independently. Welfare reduction from taxation is consistent regardless of circuit breaker status.

\begin{figure}[H]
\centering
\includegraphics[width=0.8\textwidth]{figures/baseline_governance/welfare_vs_tax_by_cb.png}
\caption{Welfare vs tax rate grouped by circuit breaker state.}
\end{figure}

\subsection{Significant Results (Bonferroni-corrected)}

\begin{table}[H]
\centering
\begin{tabular}{lllll}
\toprule
\textbf{Comparison} & \textbf{Metric} & \textbf{$d$} & \textbf{$p$} & \textbf{Survives} \\
\midrule
Tax 0\% vs 15\% & Welfare & 1.41 & 0.0001 & Bonferroni \\
Tax 0\% vs 10\% & Welfare & 1.33 & 0.0002 & Bonferroni \\
Tax 0\% vs 10\% & Honest payoff & 1.29 & 0.0002 & Bonferroni \\
Tax 5\% vs 15\% & Welfare & 1.13 & 0.0010 & Bonferroni \\
Tax 5\% vs 10\% & Welfare & 1.09 & 0.0015 & BH only \\
Tax 0\% vs 15\% & Honest payoff & 1.08 & 0.0016 & BH only \\
\bottomrule
\end{tabular}
\caption{Statistically significant results after multiple comparisons correction.}
\end{table}

All significant effects involve the transaction tax on welfare or honest payoff. No toxicity, quality gap, opportunistic, or deceptive payoff comparisons reach significance.

\subsection{Agent-Type Stratification}

\begin{table}[H]
\centering
\begin{tabular}{ll}
\toprule
\textbf{Agent Type} & \textbf{Mean Payoff} \\
\midrule
Honest & 14.50 \\
Opportunistic & 11.46 \\
Deceptive & 2.26 \\
\bottomrule
\end{tabular}
\caption{Mean payoff by agent type across all configurations.}
\end{table}

Pairwise comparisons (paired $t$-test):

\begin{table}[H]
\centering
\begin{tabular}{lll}
\toprule
\textbf{Comparison} & \textbf{Cohen's $d$} & \textbf{$p$-value} \\
\midrule
Honest vs Opportunistic & 0.84 & $<0.0001$ \\
Honest vs Deceptive & 5.73 & $<0.0001$ \\
Opportunistic vs Deceptive & 2.91 & $<0.0001$ \\
\bottomrule
\end{tabular}
\caption{Pairwise agent-type payoff comparisons.}
\end{table}

Honest agents earn significantly more than all other types. Deceptive agents earn the least, suggesting that the baseline governance environment effectively penalizes deception even without explicit audit mechanisms.

\subsection{Tax Impact by Agent Type}

\begin{table}[H]
\centering
\begin{tabular}{llll}
\toprule
\textbf{Tax Rate} & \textbf{Honest (mean $\pm$ SD)} & \textbf{Opportunistic (mean $\pm$ SD)} & \textbf{Deceptive (mean $\pm$ SD)} \\
\midrule
0.00 & $16.14 \pm 2.44$ & $11.95 \pm 4.49$ & $2.17 \pm 0.80$ \\
0.05 & $15.20 \pm 2.85$ & $12.61 \pm 5.46$ & $2.59 \pm 0.85$ \\
0.10 & $12.94 \pm 2.50$ & $11.41 \pm 3.94$ & $2.08 \pm 1.70$ \\
0.15 & $13.70 \pm 2.05$ & $9.88 \pm 2.53$ & $2.19 \pm 1.56$ \\
\bottomrule
\end{tabular}
\caption{Agent payoff by type and tax rate.}
\end{table}

The welfare reduction from taxation is borne primarily by honest agents ($16.14$ to $12.94$, a 20\% decline from 0\% to 10\% tax). Deceptive agent payoff is unchanged across conditions ($\sim 2.2$), making taxation regressive: it taxes productive behavior without reducing exploitative behavior.

\begin{figure}[H]
\centering
\includegraphics[width=0.8\textwidth]{figures/baseline_governance/agent_payoff_vs_tax.png}
\caption{Agent payoff by type vs transaction tax rate.}
\end{figure}

\begin{figure}[H]
\centering
\includegraphics[width=0.8\textwidth]{figures/baseline_governance/agent_payoff_by_type.png}
\caption{Agent payoff distribution by type (box plot).}
\end{figure}

\subsection{Normality Validation}

\begin{table}[H]
\centering
\begin{tabular}{llll}
\toprule
\textbf{Group} & \textbf{Shapiro-Wilk $W$} & \textbf{$p$-value} & \textbf{Normal?} \\
\midrule
Tax 0\% & 0.9370 & 0.2101 & Yes \\
Tax 5\% & 0.9598 & 0.5403 & Yes \\
Tax 10\% & 0.9708 & 0.7707 & Yes \\
Tax 15\% & 0.9678 & 0.7087 & Yes \\
\bottomrule
\end{tabular}
\caption{Shapiro-Wilk normality tests on welfare distributions.}
\end{table}

All groups pass normality tests ($p > 0.05$), validating the use of parametric $t$-tests.

\begin{figure}[H]
\centering
\includegraphics[width=0.8\textwidth]{figures/baseline_governance/welfare_toxicity_tradeoff.png}
\caption{Welfare-toxicity tradeoff scatter plot colored by tax rate.}
\end{figure}

\section{Discussion}

\subsection{Key Findings}

\textbf{Transaction taxes reduce welfare without reducing toxicity.} The strongest finding is that taxation imposes a deadweight loss ($d=1.41$ for 0\% vs 15\%) while toxicity remains constant at $\sim 0.305$. This suggests that in a baseline environment, taxation does not selectively deter harmful interactions --- it reduces interaction volume uniformly, affecting beneficial and harmful interactions equally.

\textbf{The tax burden falls disproportionately on honest agents.} Honest agents lose 20\% of their payoff between 0\% and 10\% tax rates, while deceptive agents are unaffected. This is consistent with the theoretical prediction that agents who generate the most surplus (honest agents, through high-$p$ interactions) pay the most in transaction taxes, while agents who generate little surplus (deceptive agents) have less to tax.

\textbf{Circuit breakers are inert in stable environments.} This is a null result, but an important one: circuit breakers are a latent governance mechanism that only activates under stress conditions. In a stable 5-agent, 10-epoch environment, systemic risk indicators never exceed thresholds. This motivates future studies under adversarial stress-testing conditions.

\subsection{Non-linearity of Tax Effects}

The welfare decline is not linear in tax rate. The jump from 5\% to 10\% (60.8 to 52.3, a 14\% decline) is larger than from 0\% to 5\% (62.5 to 60.8, a 3\% decline) or from 10\% to 15\% (52.3 to 53.2, essentially flat). This suggests a phase-transition-like threshold between 5\% and 10\% where tax costs exceed the marginal value of some interactions, causing agents to stop participating.

\subsection{Implications for Governance Design}

These results challenge the naive view that transaction taxes are a universal governance tool. In this environment:
\begin{itemize}
\item Taxes do not improve toxicity outcomes
\item Taxes disproportionately harm honest agents
\item The welfare cost of taxation is non-linear and potentially catastrophic above certain thresholds
\end{itemize}

More targeted governance mechanisms --- such as reputation-weighted taxation, audit-based penalties, or staking requirements --- may achieve better outcomes by selectively increasing costs for low-quality interactions while leaving high-quality interactions unaffected.

\section{Limitations}

\begin{enumerate}
\item \textbf{Small agent pool}: 5 agents limits the diversity of strategic interactions. Larger populations may exhibit different dynamics.
\item \textbf{Short time horizon}: 10 epochs may not capture long-term equilibrium effects of taxation.
\item \textbf{No reputation dynamics}: Reputation decay is disabled (rate=1.0), which removes a key feedback mechanism.
\item \textbf{Binary governance levers}: We test only on/off circuit breakers and fixed tax rates, not adaptive mechanisms.
\item \textbf{No network effects}: Agents interact in a complete graph; realistic systems have network structure.
\end{enumerate}

\section{Reproducibility}

\begin{verbatim}
# Reproduce the sweep
python runs/20260213-173805_baseline_governance/run_sweep.py

# Reproduce the analysis
python runs/20260213-173805_baseline_governance/analyze.py

# Reproduce the plots
python runs/20260213-173805_baseline_governance/generate_plots.py
\end{verbatim}

Scenario file: \texttt{scenarios/baseline.yaml}\\
Seeds: 200--209 (10 per configuration)\\
Total runs: 80 (4 tax rates $\times$ 2 circuit breaker states $\times$ 10 seeds)

\end{document}
